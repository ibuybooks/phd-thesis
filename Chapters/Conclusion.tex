\chapter{Conclusion}

Several methods were used to characterize the physiological state of the hamster uterus, with regard to oxygen metabolism, in two age groups, young and senescent. Most of the factors studied were consistent with the concept that the aged uteri are in a relatively ``estrogenic'' state, but this does not necessarily imply an elevated concentration of steroid estrogens. Factors such as vitamin E deficiency, age pigment deposits, porphyria, ammonia, collagen accumulation, and progesterone deficiency could promote this state, as well as possibly elevated levels of adrenal or ovarian estrogens or other hormones such as catecholamines. In senescent animals oxygen consumption was increased, TTC reduction was high (TTC reduction may simply reflect the oxidative capacity---Cascarano and Zweifach, 1955), catalase was low, peroxidase and uterine pigment were high, and uterine weight was increased considerably. Oxygen tension of the uterine lumen was extremely low in cycling animals of all ages, but was distinctly lower in senescent animals. These changes might be ascribed to a mere vitamin E deficiency, except that the requirement for vitamin E is a function of age. It is possible that the vitamin E's ``oxygen sparing'' effect may be responsible for its effect on reproduction, as it seems to be for its effect in cardiac angina. Considering the anti-fertility effects of small amounts of exogenous estrogen, it is suggested that this state is responsible for senescent infertility, and it is proposed that luminal oxygen tension may be the crucial factor in both the artificially induced estrogenic sterility and in senescent sterility. Since any irritation promotes estrogen binding by the uterus, the oxygen effect may be even more general. The extraordinarily low oxygen tension recorded in the lumens of both young and senescent hamsters---apparently lower (especially in the older animals) than any mammalian tissue that has been studied except cancer---suggests that a drastic increase of oxygen tension must be achieved around the time of implantation if the embryos are to survive. A few measurements suggested that this is the case, and a similar change has been reported in other animals, but the closure of the lumen at the time of implantation makes measurements at the time very questionable. If the equivalence of TTC reduction to QO$_{2}$ can be accepted, then a study of age differences in TTC reduction by uterine tissues at implantation sites throughout pregnancy would be a very desirable step in confirming this mechanism.