\chapter{Methods and Materials}

\section{Animals}

Hamsters (\textit{Mesocricetus auratus}) which had been maintained under conditions of controlled light periods and partially controlled temperature and housed in individual cages were used in these
experiments. They were fed lab chow pellets and lettuce; from August to December, 1971, they were given a vitamin supplement, including vitamin E. Two age groups were used: ``young,'' three to six months
old, and ``senescent,'' thirteen to sixteen months old.

When the whole uterus was to be used, the animals were given 0.5 ml of sodium pentobarbital solution (50 mg/ml) by intraperitoneal injection, and when measurements were to be made in vivo 0.25 ml/100 gr. body
weight was given. Animals were followed though at least 2 estrus cycles, using the method of Orsini (1961), and were used on day 4 of the cycle.

For enzyme studies, the uteri were removed, blotted and gently stripped to remove fluid, and weighed. Wet weight is used in all experiments, since large amounts increased in collagen would make dry weight or protein
basis misleading. The decreased water content would tend to offset increased collagen of old tissue. For the tetrazolium assay, 40 mg. pieces of whole uterus were blotted, weighed, and immediately dropped into the tetrazolium
solution at incubation temperature. When endometrium and myometrium were to be assayed separately, the uterus was opened by cutting one side longitudinally, and placed on a cool glass surface, where the endometrium was
quickly removed by scraping and placed in cold saline until weighed. The tissue was homogenized in one ml. of saline for each gram of fresh tissue.

For respiration measurements, the tissue was prepared as above except that instead of being homogenized, it was cut with small scissors into pieces about 0.5 mm in diameter. Pieces of approximately
10 to 15 mg. of endometrium, myometrium, or whole uterus were blotted on dry glass and quickly weighed, then returned to cold Krebs-Ringers before being cut. The chopped tissue was immediately placed in 2 ml. of
cold Krebs-Ringers in the respirometer flask, or in the electrode chamber.

\section{Oxygen Consuption}

GILSON MANOMETRIC METHOD: 10 to 15 mg. of uterine tissue prepared as described above was placed in 3 ml. of Krebs-Ringers solution, pH 7.1 in a manometer flask, with 0.2 ml. of concentrated
KCl solution in the center well with a filter paper wick, and allowed to equilibrate for 5 minutes at 36\textdegree{}C  before recording oxygen consuption. Readings were made at 5 or 10 minute
intervals for one hour. In some cases, concentrated sulfuric acid was added to the vessel side-arms.

BECKMAN CLARK ELECTRODE METHOD: 0.2 ml. of Krebs-Ringers solution was injected into the recording chamber and allowed to reach 37\textdegree{}C; 10 mg. of tissue, prepared as described above, was
placed in the chamber and the Beckman Clark electrode was inserted, and recording was begun immediately, using the Beckman 160 Physiological Gas Analyzer and a Varicord recorder.

\section{Luminal Oxygen Tension}

Some measurements were made using a Beckman platinum micro-electrode inserted into the uterine lumen though an 18 gauge arterial needle. Most measurements were made by cutting an
opening on the anti-mesometrial side of the uterus about 0.5 cm below the uterotubal junction and inserting the bare micoelectrode directly into the lumen. One horn of the uterus
was exposed for more than one measurement. Thirteen cycling animals were used, and two on day 4$\frac{1}{2}$ of pregnancy. Insertion of the electrode though the vagina may allow air to enter.

\section{Triphenyl Tetrazolium Chloride Reduciton}

Fourty mg. of fresh tissue was dropped into 4 ml. of a 0.5\% solution of TTC in 0.1 M phosphate buffer, pH 8.5, and incubated one hour at 37\textdegree{}C. The tissue was then removed and left
in 4 ml. of reagent grade acetone until all color was removed. As a blank control, 40 mg. of fresh tissue was placed directly into 4 ml. of acetone. The color intensity was recorded
photometrically (at 475 nm.) and the quantity of formazan present was determined by comparison with known quantities of formazan in acetone.

\section{Enzymes}

TOTAL LDH: 0.1 ml. of enzyme solution (supernatant fluid of the homogenate) was diluted with 1.0 ml. of saline and 0.1 ml. of this solution was added to Sigma Standard Substate, incubated
for 30 minutes, developed with Sigma Color Reagent, and optical density was measured in a Beckman Model B spectrophotometer at 570 nm. wavelength.

LDH ISOZYMES: Enzyme solution was prepared as above, except that 0.1 ml. of the enzyme was diluted with phosphate buffer, 0.1 M, pH 7.5, and heated to 57\textdegree{}C to inactivate the heat
sensitive isozyme 1 (liver or muscle isozyme), and another 0.1 ml. was heated to 65\textdegree{}C to inactivate all isozymes but the relatively heat stable isozyme 5 (heart isozyme). The heated
enzymes were then treated as above, and their activities were compared by electrophoresis and stained with Sigma Color Reagent and Substrate.

Special care must be taken with the homogenate for LDH measurement, since the M$_{4}$ isozyme is cold-denatured (Zondag, 1963).

$\beta$-GLUCURONIDASE: Phenolphthalein glucuronide (0.01 M) was used as a substrate, and incubated at 37\textdegree{}C in an acetate buffer at pH 4.5 for 1 hour, with 0.1 ml. of the same enzyme solution
used for LDH assays (Fishman and Green, 1956; Talalay, Fishman, and Huggins, 1946; Fishman, Springer, andBrunetti, 1948). Released Phenolphthalein was made visible by raising the pH to 10.5 with glycine
buffer, and measured in the photometer at 540 nm. Fixed sections of uterus were also treated, for approximate localization of the enzyme.

PEROXIDASE: P-phenylenediamine (0.01 M) was used as substrate, and incubated at room temperature. The photometric measurement was made every minute during the reaction. 0.1 ml. of the same enzyme solution
as above was used. Fixed sections of uterus were also stained with p-phenylenediamine for general localization of peroxidase activity.

CATALASE: Using 0.01 M hydrogen peroxide as substrate, 0.1 ml. of the enzyme solution described above was placed in a respirometer flask with 2 ml. of substrate solution (in phosphate buffered saline, pH
7.5, 0.15 M) and the rate of oxygen evolution was recorded. With the same substrate concentration, both enzymes should reach the same total amount of oxygen generated, though at different times, but the concentration
of H$_{2}$O$_{2}$ used inactivates both enzyme solutions in about the same length of time, leaving some substrate unchanged when the reaction stops from enzyme damage.

Animals to be used in NMR studies of estrogen influence were injected subcutaneously with 5 $\mu$g if estradiol in albumin solution three days before using, and ovariectomized animals were
kept in cool saline until they were placed in NMR vials.

\section{Assays}
BOUND-ESTROGEN ASSAY, BROWN-ITTRICH FLUORESCENSE METHOD: (Lunass, 1962). Whole uterine homogenate was extracted with 2 volumes of acetone, which was partially evaporated, and placed in
fuming sulphuric acid, and then extracted with 4 ml. of methylene chloride. This was placed in quartz cuvettes, and a Hitachi Perkin Elmer MPF-2A fluorescence spectrophotometer was used to scan
at 565 nm. while exciting with u.v. at 546 nm., using slits \#5, and sensitivites 5 and 6.

DYE COUPLING METHOD: (4-amino-6-chloro-m-benzenedisulfonamide). Urbanyi's method developed for pharmaceutical use was applied to a methanol extract of the uterine homogenate (Urbanyi and Reim, 1966).

BIO-ASSAY: Whole uteri were extracted in acetone (Jkubowska-Naziemblo, 1969), which was evaporated and the extract resuspended in 1\% ethanol in saline solution. Half of this was injected as 2
subcutaneous doses (1 ml. each) on consecutive days into each ovariectomized animal. Their uteri were inspected and weighed two days after the second injection.

UTERINE AGE PIGMENT EXTRACTION: Whole senescent uteri which had been cleared in benzyl benzoate and which were heavily pigmented were soaked in reagent grade diethyl ether or pyridine until
the solvent was slightly colored, about 24 hours. Ether or pyridine extracts of old uteri were scanned at visible frequencies in a Beckman Model B spectrophotometer, and at ultraviolet frequencies
in a Cary 14 ultraviolet photometer.

\section{Uterine Weight, Water Content, and Diameter}

After weighing the hole fresh uterus, pieces of 50 to 75 mg. of whole uterus, ot 10 to 20 mg. of endometrium, or 50 to 100 mg. of myometrium were
occasionally taken for dehydration and determination of wet weight/dry weight ratio. Representative slides of fixed uterine cross sections were measured to determine
overall diameter and thickness of myometrium and endometrium.

\section{Vitamin E Treatment}

Three animals were given 125 units of vitamin E once a week for six weeks, orally, and were then used for oxygen tension measurements.

\section{NMR Spin Echo of Tissue}

Recently removed uteri were examined in a Nuclear Magnetic Resonance Specialties Spin Echo Spectrometer, at 25 megahertz.

\section{Inspection of Harderian Glands}

The eyes of recently killed hamsters were pushed aside and the tissues of the eye socket were examined under long-wave ultraviolet light for red fluorescence.

\section{Uterine Age Pigment Extraction}

Whole senescent uteri which had been cleared on benzyl benzoate and which were heavily pigmented were soaked in reagent grade diethyl ether or pyridine
until the solvent was slightly colored, about 24 hours. Ether or pyridine extracts of old uteri were scanned at visible frequencies in a Beckman Model B spectrophotometer, and
at ultraviolet frequencies in a Cary 15 ultraviolet photometer.

\section{Statistics}

Differences between means were evaluated by Student;s t test, with Cochran's approximation to the Behrens-Fischer correction for samples of different variation when necessary.