\chapter{Introduction}

\section{Sex and Aging}

Theories of aging are historically closely associated with ideas about
sexuality, and it is only in the twentieth century that the two have
been habitually separated. The earliest tradition that was explicitly
concerned with the nature of aging is that of the Taoist culture of China.

In China, experimental science was generally closely related to the Taoists; the
concept of natural balance and their desire for long life or immortality of the
body. Deterioration of the body with age was considered to result from the imbalance of
various factors, including masculinity and femininity, and attempts were made to find
techniques or materials to cure disease, impotence, sterility, and aging itself. Ko Hung
(a Taoist experimentalist) believed in the early fourth century A.D. that plant drugs would
prolong life, and a compound of minerals and metals would give immortality. While other
Taoists experimented on criminals, Ko Hung used plants and animals to study aging (Needham, 1956).
Ginseng, which is rich in water-soluble ``anabolic'' steroids (Bykhovtsova, 1970; Claus, Tyler and Brady, 1970) is
the best known of the Chinese herbs used for prolonging sexual vigor and lengthening the life span. Recent
studies indicate that ginseng promotes resistance to disease and stress, and increases the life
span of rats (Yudkin, 1970; Grinevich and Brekhman, 1970). Eventually, in the Middle Ages, sophisticated
methods were developed for extracting highly purified steroids from human urine, and these were
used in treating senescence (Needham and Lu Gwei-Djen, 1968).

In the late nineteenth century, Brown-Sequard in France believed that testicular extracts could restore
sexual, physical, and mental vigor to old men. At the same time of his announcement he was ridiculed
and said to be deluded (Steinach, 1940). Up to the present time his claim is not generally accepted
(Turner, 1966; Connors, 1969) because of the supposed insolubility of androgens in water, though testosterone
is routinely used for postponing some of the symptoms of senescence in men. However, Brown-Sequard's claim seems
reasonable enough, considering the abundance of fatty acids, lipo-proteins, and soluble simple proteins in the 
testes that would be able to make steroids soluble in water. Even the slightly polar nature of the androgens themselves
(their hydroxyl and carbonyl groups) gives them a certain small degree of solubility in water.

The gonadal theory of aging (i.e., that degeneration of the gonads precedes and causes the degeneration of all other
organs: Steinach, 1940) was generally discarded by 1940, but by the same time it was widely accepted that aging does
involve changes in hormone controlled processes, and that hormonal insufficiently or imbalance is likely to be
intimately involved with senescence. Increasing biochemical sophistication was leading most investigators to look for
subtler or more basic processes as the primary source of aging. Gonadal degeneration is now more modestly considered to 
be one of many changes that occur as a result of a more basic process of aging, but its importance as a possible immediate cause of
senescent sterility in the female is the subject of current investigations, which will be discussed below.

\section{Physiological Theories of Aging}

Two other 19th century ideas have remained viable, in modified
form, up to the present. E. Metchnikof (1908) proposed that the
basic cause of aging was connective tissue degeneration; the current
version of the theory, viz., that aging results from molecular
changes in collagen and structural changes in connective tissue, is
now widely held, and fits the evidence well. He also (1908) believed
that toxins from putrefaction in the intestine caused aging, and cited
correlations between short intestines and long life spans. It has recently
been suggested that polyamines, e.g., putrescine, with histamine-like effects, may
be absorbed from the intestine and provoke reactions of an allergic type. Korenchensky
(1961) has discussed more recent versions of the ``auto-intoxication'' theory. Toxins 
of various types are known to induce auto-immune reactions (Nikolayev, et al., 1970), which
have been implicated in certain degenerative diseases and are believed by some to be a cause
of aging, that is, the primary event in aging is presumed to be the appearance of antibody
producing cells incapable of distinguishing themselves from others. Viruses and diet have been
implicated in auto-immune disease in NZB mice (Fernandes, et al., 1972). As a general cause of 
aging the auto-immune process would seem to be only a special cause of somatic mutation, and A. Comfort
(1969) has convincingly argued, in relation to Curtis' defense of the somatic mutation theory of
aging, that somatic mutations are normally not numerous enough to account for aging, and that a cause 
prior to the mutations must be sought. What at first seemed to be a cause of aging turns out to be
an effect of aging. D. G. Carpent and J. A. Loynd (1968) similarly criticize the somatic mutation theory
on the basis of the number of mutations that occur, but add that it fails to explain the effect of diet, and the
changes that uniformly appear with age. P. Alexander and D. I. Connall (1963) have shown that the powerful mutagen
ethyl methane sulphonate had no effect on the life-span of mice, in spite of causing more mutations than x-ray
treatments which did shorten the life-span of mice, and conclude ``that a non-genetic cause must be sought for the
reduction of life expectancy which occurs in CBA mice after large doses of x-rays\dots'' Although R. H. Mole (1963) has
claimed that there is ``no additive or synergistic effect of whole body irradiation by fast neutrons and natural aging,'' this
does not apply to other forms of radiation, and some data will be mentioned below that show a synergism between x-irradiation and
dehydration in producing mutations and cell death. It is also interesting to note that Brachet (Bell, 1972) produced
chromosomal abnormalities in amphibian eggs by merely heating the cortex. The evidence generally favors the idea
that both aging and chromosome damage are physiologically caused, rather than the idea of Curtis and others that aging
is produced by random somatic chromosomal mutations (Strehler, 1967).

The concept of accumulation of toxic material with age has the advantage of appearing to be a simple physical process that
correlates very well with age (Dubina, 1970) and that could reasonably be expected to cause some of the functional changes
associated with age, but again the process of metabolic detoxification, which can be assumed on the basis of various lines of
evidence to decline with age, could account for the accumulation of toxic metals (Selye, 1967) and other materials with age. This
criticism, however, is not as convincing as similar criticisms of other theories because of the peculiar vulnerability of the primary
organ of detoxification, the liver, to toxins (Song and Kappas, 1968; Biskind, 1946). A very obvious positive feedback system
must be involved, but the appropriateness of the feedback concept to a process of lifelong aging hasn't been
demonstrated, and regenerative and higher control processes would add further complications. The capacity of heavy metals to 
as catalytic centers, modifying the nature of the collagen system, suggests an overlap of the collagen and toxic theories of aging. The
inactivation of enzymes by certain metals (e.g., silver, mercury, arsenic) suggests that this theory would also have a degree of
similarity to some of the somatic mutation and cybernetic theories of aging.

The theory of somatic mutations, resulting from background radiation or other random processes that would place the mutation as the 
first event in aging, is probably inadequate, as mentioned above, but there is evidence of some kind of physical change---possibly
crosslinking---in at least some kinds of DNA, which increases with age (Hahn and Fritz, 1966). Increased melting temperature of DNA has been
offered as evidence of a physical change assumed to involve inactivation of genes, but it is known that the melting temperature of another
nucleic acid, transformer RNA, is very sensitive to ion concentration (Miller and Byrne, 1967). It might be that altered counter-ions are
responsible for the observed change in DNA with age. It is known that general tissue concentrations and ratios of ions change with age (Kohn, 1971) and
it seems that at least some of the change is intracellular (Friedman, et al., 1965). Repetition of these experiments with purified DNA might
settle this question.

An idea that used to be generally believed and that is still frequently encountered is that the nucleus is the center of all control processes
in the cell and that therefore a change of cell function requires a loss of some kind in the information stored in the nucleus; an extreme form
of the idea was the doctrine that only the germ line possessed a full complement of genes, and that somatic differentiation occurred by selective loss 
of information. Gurdon's (1962) transplanting of nuclei from differentiated gut epithelium into enucleated egg cytoplasm was a clear refutation of that doctrine, since
the nuclei allowed development into a late embryo. Although discredited for most organisms, the ideas has given moral support to at least two
theories of aging, viz., somatic mutations and Hayflick's idea (descended from Weismann's) of germ immortality - somatic mortality. Vegetative reproduction
in plants is an obvious exception to Hayflick's principle of somatic cells being limited to a small number of divisions. Hayflick attempted to discount changes
of the culture medium and ``time'' as factors inhibiting mitosis after a certain number of divisions, but time was measured with the cultures in the frozen state, which
suggests that he has a strange conception of how time affects chemical and biological reactions. His demonstration that the medium didn't become progressively inhibitory
was simply to put new fetal cells into a culture of cells that had stopped dividing, and show that they were able to divide. Strehler (1967) has shown
that the number of divisions depends on the dilution factor. What Hayflick argues for is some kind of internal clock, designed to stop after a certain number
of divisions, independent to a high degree of its environment. This internal clock concept is still held by some theorists to account for control of growth and differentiation, in
opposition to the concept that cells differentiate in response to a series of inductive and inhibitory influences, which varies with time and their position in the
organism. It has been demonstrated in Strehler's laboratory (1967) that physical restraint (an agar film) that prevented mitosis allowed some kind of aging process to occur, in
either the cell or the medium. Strehler (1967) also demonstrated, by labeling stem cells in skin, a capacity for division that is apparently unlimited, or at least is on the
order of 100 times greater than the number set by Hayflick as a limit. Skin (Krohn, 1966) and mammary gland (Daniel, et al., 1968; Daniel, 1971; Daniel and Young, 1971) transplant
experiments have been used to argue for or against the idea of somatic cell mortality resulting from a fixed number of mitoses, but the arguments have been inconclusive in both cases. (Apparently
it is generally tacitly assumed that transformed malignant cells are no longer somatic cells.) Hayflick's inhibition could be the result of either the presence or the absence of a
chemical substance, or a physical factor, such as might be involved in Auerbach's discovery of reprogramming of lymphocytes by contact with other cells. Failure to
consider the dilution factor is an important weakness in Hayflick's theory, just as it was in the (clonal selection theory) experiments criticized by Auerbach (1970).

Daniel, et al., used mammary transplants in mice to argue for Hayflick's theory, but their assumption that immune reactions are not involved is not beyond question, and the
experiments really just show that a tissue can outlive an organism. The purpose of the experiment was to study aging as it relates to an organism, but the survival of the tissue
for several generations shows that the process involved in its eventual death is not likely to be relevant to aging in the normal life span. They found, in fact, that the age of the
tissue donor did not affect the lifespan of the transplants. The real significance of the experiment is to weaken the theory of programmed cell death as a factor determining the lifespan
of an organism. The absence of an effect of donor age tends to support Kohn's belief that intracellular changes are not the most important factor in aging. More recently, Daniel, et al., (1971) reinterpreted the
results as showing only that the tissue ``gradually loses its capacity to respond by proliferating to a hormonal stimulus that is relatively constant and which continues over a long period of time.''

The inhibitory property of serum from old animals (Kohn, 1971) possibly relates to Hayflick's results, but it is very interesting from several other points of view. For example, it has been
proposed (Comfort, 1964) that the cell loss which is typical of aging is the result of inhibition of mitosis by chalones and adrenalin. According to Bullough (1971) both aging and cancer result
when inhibition is lost, and stress, by maintaining high levels of adrenalin, can prolong life. Delayed regeneration in old animals might relate to a high level of chalones. Adrenalin
concentration may be elevated in old age, and sensitivity to exogenous adrenalin increases with age (Strehler, 1967; Frolkis, et al., 1970) which is consistent with a high basal level, and with
the concept of increased mitotic inhibition with age.

D. G. Carpenter (Carpenter and Loynd, 1968), a nuclear weapons engineer engaged in systems analysis of space weapons, has proposed an integrated theory, which consists in saying that many partially
true theories of aging when taken together are more completely true. Regardless of his conclusions, his interest in feedback and amplification as part of an organism's control system is probably valid.

Sacher (1959) has proposed that the brain is the control system most relevant to aging, and the idea of neurone loss with aging is so popular that one of the most frequently repeated remarks about one's body
concerns the supposed daily loss of 100,000 brain cells. The suggestion has even been made that the loss is so selective that it could be the basis for memory (Dawkins, 1971). Regardless of the idea's popularity
and its good foundation in some insects, it certainly is not unchallenged (Comfort, 1964). Even the concept of age-related decline of mental function (e.g., the limit of learned reaction time or nonsense memorization) has
been challenged by experiments using populations with equivalent occupations or an opportunity for relearning (Murrell, 1969). Analogous to the supposed decline in mental function with age (as distinct
from disease) is Terman's observation that bright females suffer a severe loss of mentality at the age of 23, with the shift of occupation from student to housewife. The critics of the concept of brain loss with age
sometimes claim that the brain is one of the most durable populations of cells, and that the ``horizontal'' averages of brain quality include a majority of diseased individuals, that obscures the lack of effect
of mere aging (Pfeiffer, et al., 1968). However, this controversy is not necessarily essential to Sacher's idea of the brain as the most important control system in aging. Sacher's observation is that the mass of the brain
correlates very well with the length of life (1959). His belief is that the brain maintains biological equilibrium, and that with accumulated damage to the brain as a control system (whatever its nature) the organism
would become increasingly unstable and die.

Sacher's correlation of brain size with length of life (``it's an advantage to have a brain, and disadvantage to have a body'') recalls G. W. Crile's (1941) discovery of a close correlation between brain size and total
metabolic activity (oxygen consumption per hour per animal) both between and within species. Crile's observation accounts in an odd way for the old approximate ``two-thirds power'' (actually 0.7 to 0.8 for most species) rule
for increase of metabolic activity with increased body weight within a species (Florey, 1966), because this is approximately the proportion in which brain size increases with body weight (Lilly, 1963). The
well-known actuarial data that show that short people live longer than tall ones might be relevant to this question, because of their proportionately larger brain size; the same would apply to the greater
longevity of women. Energy production has been proposed as a crucial factor in life-span (Hershey, 1970; Calloway, 1971a).

Palladin (1964) has made a possibly related observation concerning ``evolutionary level'' (which in this case corresponded to brain size) and efficiency of oxidative phosphorylation. The efficiency of coupling was found
to increase with higher evolutionary level, as well as with alertness. This observation suggests an interesting secondary factor to consider in relation to ``rate of living'' and Sacher's idea--the reason for Sacher's
correlation might be metabolic efficiency, i.e., as less wasteful use of oxygen, possibly even a less destructive use of it.

The free-radical theory of aging has become very popular recently, probably partly as a result of radiation studies and partly because of the observation (Harman, 1968; Kohn, 1971) that antioxidants such as BHT added to
the diet if mice increased their lifespan by several percent. It has been pointed out by other investigators that the mice ate less of the diet, probably because of the smell of the antioxidant, and that the restricted
food intake alone could account for the increased life span (Comfort, 1972).

Restricted intake of calories has recently been found to slow the aging of collagen in rats (Everitt, 1971).

\section{Relations of the Senescent Physiology to Other Physiological Processes}

\addtocontents{toc}{\string\let\string\@pnumwidth\string\@pnumdouble}

Other studies (Lee, et al., 1952; Visscher, et al., 1952) have shown that limitation of caloric and protein intake lengthens the life span of mice, delays development of cancer in a susceptible strain, suppresses estrus, and delays
onset of both reproductive maturity and reproductive senescence. These observations will be discussed further below.

Smith and Soderwall's (1962) finding that supplementing the diet with vitamin E delayed reproductive senescence in hamsters, and similar observations in rats tempt one in this context to suggest that excess free radicals from an overactive
metabolism are the agent of alteration in the collagen, which in turn leads to accelerated functional senescence. However, several problems exist for this interpretation, for example the relation between suppression or acceleration of sexual
maturity and the onset of senescence, and also the idea that excess calories will necessarily increase the production of free radicals, which has not been established. The literature regarding free radical concentrations in living tissue
is still in an early stage, as a result of the lack until recently of equipment that could measure EPR in the presence of water. This interpretation also fails to explain the observation that the amount of vitamin E required to maintain fertility
increases steadily and drastically with age, being at 59 weeks 67 times the amount required by the 10 week old rat. Verzar and Ermini's (1970) observation of poor creatine phosphate recovery in old animals is interesting in connection with the creatinuria
and low tissue creatine levels in vitamin E deficiency (Houchin, 1942) and the restoration of creatine kinase activity by vitamin E administration (Matusis, 1971) and the fact that the x-irradiation also produces creatinuria (Kozalka and Andrew, 1972).

Besides the functional complementarity of aging and vitamin E deficiency, there is an interesting similarity in their ability to produce lipofuscin, ``aging pigment,'' or ceroid pigment (Kaunitz, et al., 1948). This pigment has been widely discussed as an example
of intracellular crosslinking, and is assumed to be a ``clinker,'' or product of metabolism that the cell is incapable of removing (Carpenter and Loynd, 1968). Florey (1966) has referred to the existence of this material in carp which are accustomed to living under
ice at low oxygen tension and suggests that it may serve as an alternate electron acceptor in the absence of oxygen. (It is supposed that unsaturated fatty acids can serve as electron acceptor in yeast
cells in the absence of oxygen (Warburg, et al., 1967).) Regardless of the accuracy of this guess concerning its functionality, Florey's observation at least points out the feature that is common to the various situations in which it appears: fetal liver (Goldfisher and
Bernstein, 1969) has relatively low oxygen tension, and the disappearance of the pigment from the liver following birth would correspond to increased oxygen tension; aged cells are increasingly embedded in collagen, and probably experience reduced oxygen tension (Casarett, 1963); in
vitamin E deficiency, there is extremely high oxygen consumption, which would tend to lower oxygen tension, and the pigment appears as a result of vitamin E deficiency (Kaunitz, et al., 1948). That the intracellular oxygen tension is lowered by the high rate
of consumption associated with this deficiency is implied by the fact that muscular dystrophy can be induced in various experimental animals as a result of a vitamin E deficiency, and it has been demonstrated that rising dystrophic chicks in a high oxygen atmosphere
suppresses development of the disease; similarly, restricted blood circulation also produces the disease, the most reasonable conclusion apparently being that low oxygen tension produces muscular dystrophy, and a vitamin E deficiency ``wastes'' oxygen at a sufficient
rate that oxygen tension is lowered in the tissue. Additional vitamin E allows animals to survive hypoxic conditions in a higher percentage than animals fed a normal or deficient diet (Telfer, 1954).

Bjorkerud (1963, 1964) has determined that there is NADH oxidase activity associated with the age pigment. NADH oxidase is, by consuming NADH, able to release one of the potential controls over fermentation (Racker, 1972) and, at the same time, can potentially reduce a large
amount of the oxygen that enters the cell. Bjorkerud's observation would seem entirely consonant with the data concerning possibly impaired mitochondrial function in old age (Shikla and Kanungo, 1970) and the observations of Shock (1963) of undiminished oxygen consumption with age
(in relation to active cell volume). This enzyme function should be elevated in vitamin E deficiency if the same system is really active in both cases. A mitochondrial NADPH oxidase is known to increase its oxygen consumption greatly in the absence of vitamin E, and \textit{in vitro} pO$_{2}$ is
rate limiting in the peroxidative conversion of polyunsaturated fatty acids to a dialdehyde by this enzyme. Estrogen is known to induce an NADH oxidase. It has frequently been observed that many cells in the senescent hamster uterus are filled with pigment granules (Orsini, 1962). These
granules apparently require estrogen and the absence of supplemental vitamin E if they are to form (Kaunitz, et al., 1948). Peroxidase, which is induced, or activated (Lucas, et al., 1955), by estrogen, and which is known to have an NADH oxidase function (Lucas, et al., 1955; Beard and Hollander, 1962), may be
involved. (Peroxidation is being widely studied in relation to both aging and cancer.) Catalase is apparently suppressed by estrogen, which suggests that peroxide, whatever its origin (it is universally associated with respiration, but it is not known how) is being spared for another function. In cancer, a condition
similar to what is proposed here for aging exists, with low NADH combined with a ``reducing environment'' (Reid, 1965).

The proportionality, mentioned above, between onset of sexual maturity and age at which reproductive senescence, and even death, occur, has been recognized as a genetic, as well as a dietary, phenomenon by L. C. Strong (1969).

A well-known idea (Lansing, 1952) is that late maternal age tends to shorten the life-span of offspring; this seems to be in doubt even in the case of rotifers (Meadow and Barrows, 1971). (It is interesting that vitamin E happens to be necessary for sexual reproduction even in rotifers.) There is a general
belief that the effect exists in mammals, too (Lansing, 1947; Carpenter and Loynd, 1968); it is generally attributed to a higher rate of mutations in the older animal, which results in slightly defective offspring. Strong, Johnson, and Rimm (1963) found that
late maternal age is associated with an earlier onset of sexual maturity in the offspring, which would be expected if the ``Lansing effect'' is true for mammals and if the reduction of life-span is associated with accelerated
sexual maturity, as in the feeding experiments. However, Strong (1969) made the seemingly paradoxical observation that prolong selection and inbreeding of the offspring of the 201-300 day maternal age group produces mice with
fewer deviant traits, with fewer spontaneous cancers, and with longer life spans than those of strains produced by younger or older maternal age selection, and the earliest maternal age selection produced the poorest longevity.

Strong has rationalized this in terms of the concept of genetic homeostasis, which he associates with heterozygosity, as illustrated by the greater variability in age of first litter among
homozygous than among heterozygous mice. He considers the 201-300 day maternal age descent to represent a genetically more stable or homeostatic group. Many observations have been made relating the
occurrence of cancer in Strong's mice to an excess of estrogen, but the evidence for the existence of high levels of estrogen is mostly indirect. Abnormal levels of estrogen in the strain could explain many of the effects
observed, through an effect \textit{in utero} (or though milk), without recourse to more complex genetic theories.

Blaha (personal communication, 1972) has expressed the opinion that hamsters probably don't have elevated estrogen levels in old age as do old rats which often have continuous estrus, in spite of the continued existence of 17-HSD which he says suggests the presence of estrogen (1972), but no measurements
have been made of their circulating estrogen level. He has found elevated progesterone levels in old hamsters (1971), and purposes that it is the cause of their ``delayed parturition.'' High levels of estrogen are considered to be responsible for the frequent endometrial hyperplasia in rabbits and have been suggested
as a cause of the similar hyperplasia which frequently occurs in menopausal women (Cowdry, 1952; Dove, et al., 1970).

Radiation or free radicals or both are sometimes believed to be responsible for the increased cross-linking of collagen assumed to be involved in the progressive increase with age
in the melting temperature and melting contraction force of collagen, but it has been found that rat tail tendons of irradiated animals do not have the characteristics of aged rats' tendons. Whatever
life-shortening effect irradiation has, and it does seem to mimic aging in certain ways, it doesn't appear to act by way of accelerated cross-linking in collagen.

B. W. Casarett (1963) proposed that radiation and other ``non-specific'' injuries lead to an ``increase in density and amount of collagenous substance interstitially and in sub-endothelial regions
of arterioles. These changes constitute a temporal advancement in the increase of the histohaematic barrier and in the development of arteriolocapillary fibrosis, which are progressive processes in `normal' aging. Eventually
these processes cause progressive reduction in number of dependent parenchymal cells due to relative hypoxia and malnutrition.'' these ideas conform to the known effects of radiation, and he traces
in detail the idea that both radiation and aging involve increasing isolation of the cells, which in itself constitutes an injury.

Brookshy, Sahinin, and A. L. Soderwall (1961) have reported that x-irradiation mimics certain features of reproductive senescence in hamsters, for example prolonged gestation equivalent to that of the senescent animal with 100 r whole body irradiation and, with all exposures used
(100 to 700 r whole-body), increased connective tissue which infiltrates and thickens uterine muscles, glandular areas, and endothelial linings.

S. O. Brown, et al., have presented (1964) data that suggests that chronic low levels of radiation may accelerate the onset of reproductive maturity in rats. R. L. Brent (1964) observed
some similarities between vascular clamping and maternal irradiation in effects on the fetus, suggesting the possibility that radiation may act partly through an effect on circulation or oxygen tension. Neonatal
``heat'' (Boling, et al., 1939) may result from hypoxia, since no hormonal bases is apparent. Estrogen and anoxia both improve resistance to irradiation (Katch and Edelman, 1964). Radiation
mimics estrogen stimulation in several other ways: vaginal cornification (Mandl and Zuckerman, 1956), uterine weight, estrus behavior, response to progesterone, etc., and appears to have its first
physiological effect on respiration (Valentini and Hahn, 1971; Hahn and Ward, 1969). Ingram and Mandl (1958) claim that x-irradiated ovaries produce more estrogen than normal ones, without pituitary intervention.

Some of the other similarities between senescence and estrogen treatment observed in various animals are:

\begin{center}
\begin{adjustwidth}{0.1\linewidth}{0.1\linewidth}
\begin{itemize}[label={}, leftmargin=*, rightmargin=0pt]
    \item Degree of corrugation and adhesiveness of luminal surface (Martin, et al., 1970)
    \item Mitotic response in both lumen and stroma (Finn and Martin, 1969)
    \item Uterine weight
    \item Radiation resistance
    \item Vitamin E deficiency effect on uterine weight
    \item Vitamin E-estrogen antagonism on age pigments of uterus and other aging symptoms (Kaunitz, et al., 1948)
    \item Catalase level (Baird and Samis, 1971; Adams, 1955)
    \item Na/K ratio of cells (Freidman, et al., 1965)
    \item Basement membrane in reproductive organs (Fredricsson, 1969; Rowlatt, 1970)
    \item Increased tube-locking of ova (Connors, 1969; Greenwald, 1959)
    \item Chronic estrus (Bloch, 1961)
    \item Levels of uterine estrogen receptor protein (Kraay and Black, 1970; Trams, et al., 1971)
    \item Failure to implant, main embryo loss (Dreisback, 1959; Connors, 1969)
    \item Tetrazolium reduction
    \item Oxygen consumption
    \item Oxygen partial pressure in lumen
\end{itemize}
\end{adjustwidth}
\end{center}

Singhal and Valadares (1969) showed that very high levels of estrogen are required to restore ovariectomized old rats to their normal enzyme levels, which implies that high estrogen
may be normal for old animals. In their mitotic response (Finn and Martin, 1969) to estrogen stimulation, however, old tissues showed greater sensitivity than did the young; residual
bound estrogen might account for this, and the time allowed after ovariectomy might have differed in the two sets of experiments. Old animals have a similar high sensitivity to catecholamines
(Frolkis, et al., 1970), which also can be interpreted as the effect of a high basal level. Catecholamines have also been associated with ``oxygen wastage'' (Raab, et al., 1962).

Metabolic rate (oxygen consumption) has long been known to decline steadily with age, although the steady state level of thyroxin in the blood (Gregerman, 1967) doies not appear to change, but
Shock, et al., (1963) found that the loss of oxygen consumption is recalculated on a cell water basis. That is, there is a decline in the number or volume of active cells with age, as the mass of
inert connective tissue increases, but there is no decrease in the rate of oxygen consumption of the remaining cells.

In spite of this evidence that oxygen consumption does not decrease generally with age, there is some evidence that mitochondria of various tissues (e.g., liver) lose efficiency, or decrease in number, or both, with
increasing age (Shukla and Kanungo, 1970; Barrows, et al., 1960). Considering the oxidase activity of age pigment, it is possible that non-mitochondrial oxidation increases as mitochondrial oxidation declines.

Loeb (1939) has found remarkable similarities between the effect of excessive estrogen stimulation and senescence on the structure of the rat uterus. Both cause progressive increase in collagen deposits.

Arvay, et al., (1971) have shown that stress accelerates the collagen changes of rat tail tendon and uterus, apparently acting through adrenal cortical hormones and estrogen.

These similarities in the effects of x-irradiation, estrogen, and senescence, suggest another experiment which found effects of x-irradiation and chemical carcinogens similar to those known
for estrogenic stimulation. F. Devik (1963) has used the reduction of tetrazolium by skin cells to compare the effects of several carcinogenic and non-carcinogenic chemicals, and x-irradiation. Non-carcinogenic
compounds caused a decreased deposition of the reduced form of the tetrazolium on the first day, while the carcinogenic compounds and the irradiation caused an increase, though
the x-rays caused a more persistent increase. Damage to the mitochondria has been proposed as a common effect (Warburg, 1969; Devik, 1963), but since it is generally believed that tetrazolium reduction is
an indicator of dehydrogenase activity, all that is clearly indicated by these experiments is that carcinogens and radiation both activate or induce the dehydrogenase, or make more
oxidizable substrate available. The increase is very rapid. Estrogen has very similar effects (several carcinogens are known to be estrogenic in the ovariectomized female [Turner, 1966; Needham, 1942]).

The fact that age, excess estrogen treatment, and irradiation all stimulate excessive collagen deposition, are sometimes associated with cancer development, can cause increased tetrazolium reduction, and produce
similar effects on pregnancy, suggests that a new interpretation of the collagen theory of aging might be desirable. The enzyme which synthesizes vitamin C, which is important in the development of collagen, loses 
activity with age, is inactivated by lipid peroxides and preserved by adding vitamin E, and also appears to be suppressed by estrogen (Patnaik, 1971; Chaterjee and McKee, 1965).

The failure of x-irradiation to increase cross-linking in collagen tends to weaken the free-radical theory of aging, and the mutation theory cannot explain satisfactorily either the generalized cross-linking or the
generalized deposition of collagen. A common event, such as a metabolic pathway change, or a general interference with enzyme control systems, seems to be a more suitable alternative
locus for the aging process, and it would seem especially suitable if it were such as could be simply implicated in a self-sustaining (positive feedback) loop, tending to prevent reversal. For
example, collagen accumulation could increase collagen synthesis or block its degradation. Casarett's scheme was of this kind, but he relied on the mere assumption that a non-specific injury would
lead to increased density and amount of collagenous substance. What is lacking is information regarding the control of collagen synthesis, that could relate with each other the many specific stimuli that
are known to accelerate collagen formation. Chvapil, et al., (1970) found that in granuloma and in some organs in rats, prolonged hypoxia stimulates collagen synthesis and inhibits the synthesis
of noncollagen protein, and enhances mitosis in fibrotic cells. Hypoxia seems a likely common factor for the various non-specific injuries and provides a very simple positive feedback system.

A view of ionizing radiaiton's biological effects presented by Webb (1965) offers a radically different kind of action that would be at least potentially applicable to the intracellular control theory
of aging proposed by Comfort and others (Comfort, 1969). Several Russian laboratories have proposed a similar mechanism to explain the biological activity of relatively low energy microwaves (Sharp and Paperiello, 1971). Webb
found that, in bacteria, both the lethality and mutagenicity of x-irradiation depended on the relative humidity of the atmosphere in which the experiments were performed, when the
bacteria were suspended in an aerosol. He showed that partial dehydration increased the bacteria's susceptibility to both effects. Various chemicals, such as inositol, that are known to protect proteins, cells, and organisms (Troshin, 1966) from
destruction by various agents, were able to offset the effects of dehydration. Webb interpreted this  effect in terms of the supposed ability of the molecules' hydroxyl groups to stabilize bound water which in turn would
stabilize macromolecules, or larger structures such as membranes. A mechanism by which the pathway changes induced by estrogen, radiation, anoxia, and aging might result from a more or less direct action on intracellular
water has been discussed elsewhere (Peat and Soderwall, 1971, 1972).


Alteration of the solvent properties of cell water could have very far reaching effects on the function of an organism (Atkinson, 1971).

One of the factors, if not the only one, that shows a universal correlation with age, even before a many-celled stage is reached, is the concentration of water (Needham, 1964). Calloway (1971a, 1971b; Calloway and Kujak, 1966) has, from an
analysis of the literature, shown that the water content of all organisms fits the same exponential curve on an absolute time scale. Thus, the tissues of a 24 month old rat contain roughly the same
percentage of water as the tissues of a 16 month old human baby (both being about the same age from conception).

Intracellular ratios of sodium to potassium tend to increase with old age (Freidman, et al., 1967). However, the opposite occurs during early development, approximately parallel to the decline in water concentration (Hazelwood, 1971). If Hazelwood (1971) and
Wiggins (1970) and Szent-Györgyi (Baird, et al., 1957) are correct in their theory of the control of the salt ratio, then one must assume that the intracellular water concentration, as opposed to the total organismic water
concentration, would increase with age. This would be compatible with the known increase of mass and compactness of collagenous tissue with age, and with the data on the decrease of
extracellular body water (thiocyanate diffusion space) with age (Kohn, 1971).

The conventional methods of measuring extracellular space have been criticized as over-estimating the true volume, at least in certain cases (Ling and Kromash, 1967). Whole organ concentrations of
calcium, sodium, and chloride (Kohn, 1971) increase with age, and there are decreases in potassium, magnesium, and phosphate; Kohn interprets these changes as reflecting
increasing extracellular space and decreasing cell volume, which accords with the microscopic picture, but there is also the likelihood (Friedman, 1967) that shifts of the same nature occur in the cell water itself.

If it is true that the intracellular water concentration is increased with age, it represents another feature of similarity among old cells, cancer cells, and estrogen
stimulated tissue, since the others are already known to have a high water concentration (Hazelwood, 1971; Damadian, 1971).

\section{The ``Estrogenic'' State}

According to Pincus and Kirsch (1935) excess estrogen \textit{in vivo} kills rabbit embryos at the blastocyst stage, but does not interfere with cleavage. Epinephrine prevents implantation in the rabbit, and hypoxia
has been suggested as the cause (Auletta, 1971). The few measurements that have been made suggest that the increase of pO$_{2}$ in early pregnancy is delayed or inadequate in the senescent hamster, but since abnormally low pO$_{2}$ has been established
in the cyclic senescent hamster, no special failure of a control system need be invoked to explain the difference at the time of implantation. These results suggest that it would be desireable to follow PO$_{2}$ levels thoughout pregnancy, and especially
at the time Thorneycroft and Soderwall (1969) observed increased resorption of embryos. Schultz (1967) has observed metabolic changes in old rats at day 12 during the period when resorption of embryos by old mothers occured, which he interpreted as a shift
toward estrogen dominance. Reduced sensitivity to deciduoma induction in rats has been attributed to high endogenous estrogen levels (Horikoshi and Weist, 1971), and Schultz has also proposed high estrogen levels as the cause of low fertility in rats selected
for small body size (1968) or high uterine metabolism (1965, 1966).

The literature provides many kinds of evidence supporting this interpretation, in spite of the popularity of the idea of estrogen insufficiency and, in humans and sometimes in experimental animals, of ``estrogen replacement theory'' for menopause. The concept of
such therapy has been challenged, but probably the strongest reasons for holding the belief in estrogen insufficiency are a) elevated gonadotropins in senescence, b) decreased urinary estrogen excretion, and c) reduction, in some species, of the number of active follicles. In
hamsters, the last reason is not relevant, since the number of follicles is normal (though the number of corpea lutea may be reduced [Thorneycroft and Soderwall, 1969b]), but Steinach's (1940) view might be relevant for some species, viz., that interstitial cells multiply
and increase hormone production whenever the germ cells are destroyed, e.g., by heat or radiation (Kennedy, 1970; Grindelund and Folk, 1962). Brooksby, Sahinin, and Soderwall's (1964) study of the effects of x-irradiation on hamster ovaries is consistent with this idea. Thung
and Hollander (1967) have expressed a similar view: ``In the ovary, the declining number of follicles is matched by the gradual evolution of proliferative cell groups\dots These hyperplastic changes in senile ovaries are found in various species, but have been studied extensively in mice\dots'' Some
of the cells ``\dots look very similar to the oestrogen-producing interstitial cells of the ovary. Actually, the occurrence of these cells in old ovaries is found to correlate with signs of increased oestrogenic stimulation.''

The other reasons given as support for the idea of estrogen insufficiency are subject to various interpretations. The evidence in support of the ovary-pituitary feedback system, existence of which is the assumption behind the argument that elevated gonadotrophins imply ovarian and estrogenic insufficiency, has
been reviewed (Borth, 1967) and it was concluded that the evidence is fragmentary and weak. However, accepting the existence of such a control system at certain times, there is evidence suggesting that it may be inoperative in senescence, that is, that adrenal
estrogens (Vinson and Jones, 1964) or ovarian estrogens (Mestwerdt, et al., 1972) may be overproduced in senescence, or adrenal estrogens may be overproduced when ovaries are removed (Diczfalusy, et al., 1964). Pineal involvement has been suggested as a factor in prolonged
estrus in old rats (Meyer, et al., 1961). Aschheim (1964) suggests that the hypothalamus loses sensitivity to circulating steroids in senescence, which would be consistent with the effect if a vitamin E deficiency on nerve tissue, and possibly with the observation of P'an, et al., (1949), that
pituitary extracts of vitamin E deficient rats caused greater gonadal stimulation in rats on a normal diet, that did extracts from rats on a normal diet. Drummond, et al., (1939) made similar observations. Orsini and Schwartz (1966) found that senescence apparently involves failure
to release LH from the pituitary. Vitamin E has been considered to be anti-estrogenic (Shute, 1940; Evans and Burr, 1927a, 1927b, 1927c, 1928), as have various B vitamins (Biskind, 1949) and vitamin A, possibly by an effect on the liver. Since the amount of vitamin E needed to prevent symptoms
of a deficiency increases greatly with age (Fuhr, et al., 1944; Emerson and Evans, 1940; Ames and Ludwig, 1964), most animals on a normaml diet eventually show some signs of a vitamin E deficiency. Since kidney resorption of the glucuronide of estrogens appears to be about zero (based on ratios of
serum urinary estrogen conjugates in preganncy), a small decrease in liver (or kidney, or other conjugating tissue) activity would tend to produce great increases in the amounts of unconjugated estrogens retained in the body. These estrogens would be almost entirely bound to proteins and lipids, but the
equilibris of protein-estrogen binding are such that almost all of the body;s unconjugated estrogens would be bound in the uterus, with smaller amounts in the vagina and mammary glands. Therefore, decrease of urinary conjugated estrogens (Poortman, et al., 1971) may directly reflect an increase of tissue-bound
estrogens. Estrogen induces the estrogen binding protein (Kraay and Black, 1970), and post-menopausal women probably have a higher concentration of the binding potein than younger women (McGuire, et al., 1972; Trams, et al., 1971). Post-menopausal bleeding in women, often assumed to result from estrogen
insufficiency, has ben found to be associated with increased estrogen activity (Frampton, 1966). However, irritation, such as an intra-uterine object causes, is also known to increase estrogen binding and uterine weight (Labhsetwar and Perser, 1972). Other estrogen-like effects of intrauterine irritation include luteolysis
(Chatterjee, et al., 1971), increased oxygen uptake (Kar, et al., 1965, 1966), increased alkaline and acid phosphatase (Karkun, et al., 1969), and uterine weight (Malajiya, et al., 1970).

It is now generally accepted that senile estrus in rats (Meyer, et al., 1961; Bloch, 1961) and cystic hyperplasia in old rabbits (Burrows, 1949) result from excessive estrogen in old age. Some old mice have a prolonged proestrus (Thung, Boot, and Muhlboch, 1956). Cowdry (1952) has discussed this as a possibility in human cystic hyperplasia
of the endometrium and other genital diseases which become very frequent after menopause. Dove, et al., (1970) have shown that urinary estrogen levels in post-menopausal women do not provide an index of the degree of estrogenic stimulation on the cytological level. However, Ward and Thyssen (1970) found that an elevated ratio of estrogens/androgens correlates
with persistent karyopyknotic smears in post-menopausal women. Progesterone insufficiency has been suggested as the cause of menopausal disturbances and senescent sterility (Poulson, 1970; Matthew, 1949; Jones, 1970; Thorneycroft and Soderwall, 1969). Insufficient progesterone causes both failure to implant and congenital abnormalities, depending on the
amount of progesterone given (Poulson, et al., 1965). Progesterone has been used in attempts to offset the anti-fertility effects of estrogen in rats, with conflicting results (Martin, 1962; Edgren, et al., 1961), and in cows, with good results (Otel, et al., 1968). Progesterone can even improve fertility are irradiation, which mimics estrogen (Valentini and hahn, 1971). Even
though it is generally accepted that, withing certain limits, it is the extrogen/progesterone ratio which is important, rather than the absolute level, one would expect progesterone to be incapable of balancing the effect of a certain amount of estrogen because of its solubility or binding properties. The effective progesterone/estrogen ratio is sometimes placed
as low as 2000:1, but several studies have suggested thta it may be as much as 300,000:1 (or even higher) (Challis, et al., 1971; Nutting-Meyer, 1970), in which case it is obvious that the capacity of a issue to absorb progesterone could easily be exceeded before reaching the concentration needed for the proper ratio. (Low solubility of progesterone has probably led to
errors in protein binding assays: Clark and Gurpide, 1972).

In spite of Thorneycroft and Soderwall's suggestion, based on the size, number, and growth of corpea lutea in senescent hamsters, Blaha (1971) found serum progesterone levels to be elevated in old hamsters. However, without data on estrogen levels, this can be considered as possibly only a compensatory response to elevated estrogen. Leavit and Blaha (1970) attribute prolonged
gestation (observed by Soderwall, et al., 1962, in senescent hamsters) to excess progesterone, though its supposed inhibitory effect on uterine contractability. Since the development of the embryo is retarded in older animals (Parkening and Soderwall, 1972; Connors, 1969) even before implantation, Blaha's theory seems inadequate to explain prolonged gestation. \textit{In vitro} studies
have shown permanent damage to the embryo from excess estrogen, but not from excess progesterone, yet both can interfere with cleavage (Daniel, 1964).

Glycogen deposits are diminished or at least delayed in their appearance in the decidual cells of senescent hamsters (Connors, 1969). This is usually attributed to estrogen insufficiency, but combined progesterone and estrogen appear to be responsible for increased glycogen deposition. Estrogen is known to activate glycogen phosphorylase (Takeuchi, et al., 1962), which under known intracellular
conditions (that is, a ratio of Pi/glucose-1-phosphate of more than 3:2, Reithel, 1967) accelerates the conversion of glycogen to glucose. Glucose or oxygen deficiency (Mitchell and Yochim, 1968) or rapid growth (Needham, 1942) leads to disappearance of glycogen. Since sufficient progesterone is necessary for decidual cell differentiation to occur (Orsini, 1962) and excessive estrogen will
entirely suppress the decidual cell response (Finn and Emmens, 1961), it seems that the delayed and diminished deposition of glycogen in the senescent animals' decidual cells should be attributed to a relative progesterone insufficiency (Pous and Comas, 1959), or even to excessive estrogenic stimulation. The higher sensitivity of the senescent uterus
to induction of mitoses in ovariectomized rats (Finn and Martin, 1969) would be consistent with an excess of bound estrogen. However, vitamin E deficiency may also prevent glycogen accumulation (Butturini, 1949).

While progesterone (and the chemically related androgens) seem to have no toxic effects, organismically or cytologically, estrogen has many deletrious effects when its concentration is too high, for example on the adrenals (Kimeldorf and Soderwall, 1947), on liver mitochondria (Gonzales-Angulo, 1970), on deciduoma formation (Finn and Emmens, 1961), on ova, \textit{in vitro} (Kirkpatrick, 1971), on liver (Song and Kappas, 1969; Biskind, 1949), on
fertility (Smith, 1926), and as a carcinogen (Needham, 1942). Korenchensky (1961) listed depression, salt retention, renal degenerative changes, hypercalcaemia, hyperlipaemia, and epithelial hyperplasia as other toxic effects.

Estrogen treatment is known to be able to cause liver damage, and prolonged treatment with estrogen lowers the activity of the ``mixed function oxidases'' (Song and Kappas, 1968; Biskind, 1946). Chronic estrogen treatment lowers UDP - glucuronyl-transferase activity (which is important for removal of estrogen
from the body) and also lowers the activity of -glucuronidase, after first raising its level (Wakabayaski, 1971). Many liver toxins can lower the liver's capacity to inactivate estrogen (Biskind, 1946). Since urinary estrogens may be 1000 times more concentrated than those in the blood, the importance of conjugation
in the liver for regulation of hormone concentration is obvious. Nutritional deficiencies are able to interrupt the liver's conjugation of steroids (and other substances), and vitamin E deficiency if known to be involved in various liver diseases (Schwarts, 1949). Since the requirements for vitamin E increase with age (Fuhr, et al., 1949), liver
function would tend to become increasingly precarious, o the extent that it depends on the vitamin. Most of the measurable symptoms of vitamin E deficiency have a precipitous onset (Fitch, 1968), appearing and becoming extreme within a few days or a few weeks, though serum tocopherol may not have been detectable for many months. Such
a physiological process might be involved in the sudden ``menopause'' when it involves many large physiological changes, but a sudden onset of infertility without other sudden changes would suggest that a simple threshold of survival conditions of the embryo may have been reached.

Estrogen seems to lower the intracellular K/Na ratio, possibly causing the cell's ``trans-membrane'' potential to become lower (Spaziani and Suddick ,1967; Jones, 1968, 1972; Goodland, 1953). Such a change in salt ratio (Orr, et al., 1972) and in electrical potential (Cone, 1970) apparently is very closely involved with the
regulation of DNA and RNA synthesis and with mitosis. Oxidative phosphorylation (Strickland, et al., 1955) is apparently necessary for maintaining high intracellular K/Na ratios, though possibly not for maintaining the resting potential (Ling, 1962), so it is possible that estrogen might have its early action either on the cytoplasmic
structure (Nemetchek-Gansler, 1967; Peat and Soderwall, 1972), i.e., producing a permability or phase change, and thus affecting oxidation by modifying structural relationships and diffusion rates and solubilities, or an oxidative phosphorylation, with a phase change (i.e., a salt ratio change) resulting from a reduction of the energy
charge. Merely incubating uterus in Eagle's medium mimics estrogen action (Mueller, et al., 1958), and hypo-osmolarity duplicates some effects (Adams and Haynes, 1969). Progesterone has an opposite effect on sodium retention (Tomlinson, 1971; Jones, 1972). Sex hormone effects seem to be universal and specific; even yeast ``sexual'' types are
selectively affected by the mammalian hormones (Takao, et al., 1971; Yanagishima, et al., 1970).

The best known scheme for estrogen action (Jensen, et al., 1968; Mueller, et al., 1958) suggests that estrogen binds to a cytoplasmic ``transport'' protein and then enters the nucleus where it modifies the chromosomes so that certain mRNAs are transcribed, leading to synthesis of new proteins which when produce all other effects by membrane and
enzymic changes. This theory ignores the fact that lipids are the first substances (by several hours) to show net synthesis (Mueller, 1958), and the observations that salt ratio changes are sufficient to initiate, stop, or modify synthesis of DNA and RNA (Allfrey, et al., 1964; Lezzi, 1969). Engel (1970) suggests that certain enzymes hav estrogen
affinity sufficient to make them ``receptors.'' Although the 9.5S and 4-6S ``receptors'' are not known to have enzyme activity, this may simply reflect the fact that the leading investigators of these proteins are not interested in their potential enzyme activity.

\section[The Problem of Senescent Sterility and Oxygen\\Wastage as its Immediate Cause]{The Problem of Senescent Sterility and Oxygen Wastage as its Immediate Cause}

Several of the best established events in the aging process---changes in collagen, water, salts, pigment, and certain enzymes---form a constellation, at least part of which reappears in several other physiological states, which may have useful similarities to aging: vitamin E deficiency, low oxygen tension, estrogen stimulation, and raditaion damage. (Muscular
dystrophy, and possibly even myocardial ``oxygen wastage,'' may also be related conditions.) Two of these states---vitamin E deficiency and radiation damage, have been well studied in their relation to reproductive senescence. This thesis will pursue the concept that low oxygen tension is a probable immediate mechanism in reduced fertility, and that the
above mentioned constellation of factors, possibly initiated by estrogen, but also a self-stimulating process common to other physiological states mentioned, is a prior cause of reproductive senescence.

Senescent hamsters ovulate seemingly normal eggs in normal numbers. There is a much greater loss of embryos at the time of implantation among old hamsters that among the young (seven times greater, accoring to Thorneycroft and Soderwall, 1969a). Some loss does occur by later resorption (two times greater than young: Thorneycroft and Soderwall, 1969b), and some
ova seem to be delayed in passage though the oviduct in the senescent animals (Connors, 1969), but their greatest difference from the young is at implantation (Connors, Thorpe, and Soderwall, 1971). It has been fairly well established in several species (Adams, 1965; Finna and Martin, 1969; Talbert and Krohn, 1966; Biggers, 1969; Maurer and Foote, 1971; Finn, 1970) that
the aged uterus or its hormonal environment is more likely to be the cause of sterility than the aged ova. Lack of adhesiveness and lack of nutritive support are two possible aspects of the uterus that have been considered as causes of aged sterility. Although differences of surface structure and content of glycogen and certain enzymes ae known, it is difficult to argue
that these are decisive. For example, even if glycogen stores are low, blood glucose should provide a substitute; if estrogen dominance leads to rapid consumption of glucose by the uterus, lactate should be available. Blaha (1964) found early embryos apparently adhering to the uterine epithelium in senescent hamsters, and only later did they disappear. The requirement of the
embryo for molecules of three or four carbons at the two cell stage disappears by the blastocyst stage (Brinster, 1971; Biggers, 1971). However, there is a large increase in oxygen consumption by the embryo at the blastocyst stage, both in rate and in total quantity. (This seems to be true for
all vertebrates.) Needham (1931, 1942) reviewed the work of Warburg, Loeb, and others on the relation of pO$_{2}$ to respiration and development, and for the organisms considered, development was retarded and respiration inhibited by moderately low pO$_{2}$. In the normal animal, according to several studies in
rats (Yochim and Mitchell, 1968; Yochim, 1971), the luminal pO$_{2}$ increases sharply at this time, just preceding implantation, and remains high as far into pregnancy as measurements have been made. This coincidence of increasing demand and increasing supply suggests that a failure of supply may be critical, since no substitute for
adequate oxygen is known. Glenister's (1970) observation that embryos in tissue culture will develop farthest in an atmosphere of 30\% to 50\% oxygen suggests that oxygen may be the limiting factor in implantation. He found that implantation occurred \textit{in vitro} irrespective of the physiological stage of the endometriums or the presence
or absence of added hormones. Glenister found that hamster blastocysts would not attach to hamster endometrium which was arranged in the form of a ``well,'' which suggests that the slightly greater diffusion distance in this configuration (which was unnecessary when he used larger rabbit embryos) might have been responsible for the
failure to attach. It is known that the shape of the vessel can govern the rate of blastocyst development \textit{in vitro} (Winterberger, Torres, and Daniel, 1971). Some of the embryos in several species are retarded in their development in old animals. With even a moderate oxygen consumption rate, the pO$_{2}$ drops sharply with the distance the oxygen must diffuse
(Longmuir, 1958). The apparent Km for oxygen of cytochrome oxidase is high, but the actual Km of the enzyme is almost infinitely small (Boag, 1970). The apparent Km for oxygen of intact cells has been found to increase with the diameter of the cell (Longmuir, 1958) and if this rule applies to ova and blastocysts, then the Km of these must be
very high: possibly close to the pO$_{2}$ of capillaries.

According to Yochim's calculations, the uterine blood flow does not increase as much as uterine respiration and uterine mass following estrogenic stimulation, with the result that the tissue becomes relatively hypoxic. The increased proportion of connective tissue (Brooks, Sahinin, and Soderwall, 1964) would probably have a similar effect. Parkening and Soderwall
(1972) have presented evidence for impaired capillary permeability in old hamsters at the time of implantation, and suggest that this may be responsible for delayed development of the blastocyst and its failure to implant. Damage to uterine circulation prevents implantation in mice (Senger, Lose, and Ulberg, 1976; Wigglesworth, 1964; Bruce, 1971).

With these considerations, it would seem appropriate to look for the causes of senescent loss of fertility in the physiological differences that exist between the young and the old uterus. The cyclic uterus would reveal basic differences that exist as a function of age, independent of the effect of embryos (McLaren and Menke, 1971). (Embryos
at different stages of development might make the meaning of pO$_{2}$ much more complex. Even with only the electrode consuming O$_{2}$ at the luminal surface or in luminal fluid, the problem of a local gradient and different diffusion rates is a possible source of error.) Besides oxygen tension, related factors will be investigated, including
rate of oxygen consumption, reductive activity, and some specific enzymes that are relevant.