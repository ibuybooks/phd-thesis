\chapter{Discussion}

\epigraph{\centering ``Every statement implies a metaphysic.''}{-- A. N. Whitehead}

\section{The Paradigm Problem}

Even seemingly simple and ``straight-forward'' problems and solutions are ultimately questions of probability, and interpretation is present in the most ``objective'' treatment of a question (Tart, 1972). A more complex problem requires a more complex context if a similar degree of ``objectivity'' is to be maintained. Whenever a paradigm is left, the question of interpretation becomes more apparent.

This thesis begins by questioning the paradigm of ``aging as decline'' of hormone concentration and metabolic rate. That paradigm was based on certain gross and organismic observations, and is valid with with regard to androgenic, luteal, and certain adrenal hormones, at least in most mammals, and with regard to whole animal oxygen consumption per unit of tissue, and probably even with regard to mitochondrial respiration. However, it lacks generality. Highlights from a large literature regarding the physiology of aging are presented in the introduction, to show that a contrary, and much more general, paradigm exists, which can be stated approximately as ``in aging and other distress conditions, typical shifts of metabolic pathways occur.'' Measurements that only seem peculiar when considered in the former paradigm, may be definitive in the latter. The typical shifts of metabolism include:

\begin{center}
\begin{adjustwidth}{0.1\linewidth}{0.1\linewidth}
\begin{itemize}[label={}, leftmargin=*, rightmargin=0pt]
    \item a)~``Damaged respiration'' in the sense of a decreased efficiency of phosphorylation, or excessive hydrolysis of ATP, by high ATPase activity, which can result from mitochondrial damage;
    \item b)~Compensatory glycolytic phosphorylation;
    \item c)~Given mitochondrial damage and NADH production in glycolysis, a necessary step in compensation is the oxidation of NADH, for example by LDH, or peroxidase, of ``age-pigment-related NADH oxidase.'' Either lactate formation, or oxygen consumption, or both, or another kind of oxidation, must increase if glycolysis is to proceed.
\end{itemize}
\end{adjustwidth}
\end{center}

This paradigm forms an interpretive link between the various ``distress conditions,'' and suggests that low oxygen tension will be a factor in these various conditions. The literature on mammalian loss of fertility under many conditions, e.g., x-irradiation, estrogen treatment, adrenaline treatment, senescence, vitamin E deficiency, progesterone deficiency, intrauterine devices, etc., is remarkable for the great similarities that exist in such diverse conditions. Estrogen action is the most common and easily manipulable of these conditions, and so each of the other conditions tends to be described as ``estrogenic.'' Knowledge regarding the states in which estrogen may be bound within the uterus is not adequate to provide a sound bases for its extraction and unambiguous assay. Whether these conditions that prevent gestation actually increase estrogen concentration in the uterus, or merely mimic it, is still debatable. The available metabolic information about the various ``distress conditions'' indicates that oxygen wastage is likely to be a common factor; data on embryos' requirement for oxygen \textit{in vitro} show that oxygen deprivation \textit{in utero} could be responsible for pregnancy wastage. The new paradigm suggests the hypothesis of loss of fertility through excessive reduction of oxygen, as a normal feature of aging, as well as in the various other ``distress conditions.'' All aspects of the hamster uterus that have been measured are consistent with this. The old paradigm does not unambiguously predict any of the observed age changes. To make it do so requires a complex series of \textit{ad hoc} assumptions, each one to explain away as ``non-estrogenic'' a change which to this point had been considered to be estrogenic. One clear prediction that it makes is that the senescent uterus will \textit{not} have an elevated metabolic rate. This is refuted by the observations, especially the TTC reduction.

Another clear prediction made by the old paradigm is that senescent sterility and estrogen-induced sterility will not act by the same mechanism, since senescent sterility is conceived of as an effect of estrogen deficiency, exactly the opposite of an estrogen excess. However, the data shows that senescence lowers oxygen tension, just as estrogen treatment does.

It is remarkable that, in spite of \textit{in vitro} studies that show sensitivity to of mammalian embryos to changes in oxygen tension, the issue has been almost entirely ignored in studies of fertility, and has never previously been considered in relation to senescent sterility. Even percentages of oxygen higher than atmospheric can be limiting to embryos implanting \textit{in vitro}, suggesting that implantation is especially sensitive to oxygen deficiency. Implantation is also the main point of damage in the various ``distress conditions,'' as far as the question has been investigated.

A paradigm by definition is open-ended enough to allow for further development, and so can never be overturned by data or reasoning, but it can become increasingly untenable. The smaller question, however, of whether oxidative changes are adequate to account for embryonic retardation, non-implantation, and resorption can be more decisively answered, because the less complex question calls on a less complex interpretive context.

The observations require rejection of the view that an oxygen deficiency cannot occur in the uterus from common physiological causes. Hypoxia has been proposed as the cause of abortion following vasoligation, but it has been generally believed that only such gross interference with circulation could produce sufficient hypoxia. That the cause can be physiological is indicated by the increased metabolic rate.

Beyond this, the observations characterize the senescent uterus as being in a very common physiological state, which can be called a ``distress state.'' The causal relation between age and the other states is discussed on the basis of the literature. Senescence is physiological, and the physiology is such that sterility must result.

The correspondence of the changes in the senescent uterus to changes in the other ``distress states'' supports the view that the changes are patterned, paradigmatic, coherent, and meaningful metabolic changes. Although the pathway details are not completely known, the existence of compensatory glycolysis and non-mitochondrial oxidation have been well established in various tissues.

\section{The Senescent State}

There is considerable evidence that the estrogen/progesterone ratio, or estrogen/17-keto steroid ratio, increases with age, at least in mice, rats, rabbits, and women, and that this change has metabolic effects similar to treatment with exogenous estrogen. However, irritation, low thyroid, excessive copper absorption, etc., probably could increase the estrogen effect without necessarily affecting the concentration of circulating estrogens. To find in senescence the metabolic state associated with estrogen treatment is interesting more for the possible effects of this state on functions such as implantation, than for its corroboration of a suspected excess of estrogen. The generality of this ``distress state'' of metabolism is intrinsically interesting.

\section{Enzymes}

Many enzymes, including ATPase, spermidine synthetase, alkaline phosphatase, NADH-oxidase, $\beta$-glucuronidase, sulfatase, peroxidase, hexokinase (Gross, 1961), malate dehydrogenase (Wilson, 1969), glucose-6-phosphate dehydrogenase and 6-phosphogluconate dehydrogenase (Schmidt, et al., 1967), lactic dehydrogenase, and ornithine decarboxylase (Kaye, et al., 1971), change in activity under the influence of estrogen, but a few are especially important in relation to oxidative metabolism. An apparent increase of cytochrome oxidase following estrogen treatment turned out to be peroxidase (Lucas, et al., 1955).

LDH was considered to be an especially suitable enzyme to study in relation to implantation because its total activity and isozyme composition are known to reflect oxygen concentration and degree of estrogenic stimulation, and to relate to the tissue's disposition of lactate and pyruvate (Johansson, 1966; Battelino, et al., 1971; Goodfriend and Kaplan, 1964; Dawson, et al., 1964). The enzyme can also function as a potential ``transhydrogenase'' system (Sanwal, et al., 1970; Young, 1961), and thus is in a position to regulate the glucose-6-phosphate oxidation pathway of carbohydrate metabolism. Also, the equilibrium

\[
  \ce{lactate + NAD+ <=> pyruvate + NADH + H+}
\]

\noindent implies that the reaction proceeds toward lactate formation, the pH will tend to rise. In many tissues the ratio of lactate/pyruvate is much smaller than the chemical equilibrium, probably because of oxidative competition for the NADH (Olson, 1963). Restricting oxygen would probably allow the reaction to move farther toward equilibrium, with a consumption of protons. Elevated pH tends to weaken other controls (e.g., the inhibiting effect of ATP, involved in the Pasteur effect) over phosphofrutokinase (Racker, 1972) which is an important control point in glycolysis. Waddell (1969) has pointed out that the addition of glucose to cancer cells tends to raise the intracellular pH, as the cells form large amounts of lactate. High pH has been observed to increase the rate of lactate formation without increasing respiration (Tobin and Mehlman, 1971). In aged rats, intracellular pH may be elevated (Comolli, 1971) as well as lactate production in certain tissues (Angelova-Gateva, 1968). The NAD$^{+}$/NADH ratio also controls the rate of glyceraldehyde-3-phosphate oxidation (Racker, 1972). LDH is thus in a position to exercise extensive control over glycolysis, possibly providing a mechanism for interfering with the Pasteur effect (which is decreased under the influence of estrogen: Greenstein, 1954). The LDH isozyme ratio may be of importance not only for determining the availability of pyruvate for oxidation, but for regulating other pathways. (Considering the low pO$_{2}$ of senescent or estrogen dominated hamster uterus, and the requirements of spermatozoa for oxygen [Nevo, 1965], the X isozyme in male producing spermatozoa [Stambaugh and Buckley, 1970] may relate to the lower proportion of males in senescence or estrogen treatment [Hahn and Hays, 1963]).

In the presence of very active NADH-oxidase, pyruvate and lactate might also have a ``coenzyme'' function in relation to this enzyme. Peroxidase, which is ``induced'' by estrogen, has a NADH-oxidase function (Lucas, et al., 1955), and this activity is also found in association with age pigment (Strehler, 1967).

There is a slight decrease of total LDH activity in old animals, probably not significant considering their higher proportion of connective tissue and collagen.

Oxygen tension regulates the isozyme ratio, both \textit{in vitro} and \textit{in vivo} (Clark and Yochim, 1971; Yochim and Clark, 1970; Fujisawa, 1969; Johansson, 1966). Citric acid cycle intermediates can retard the formation of isozyme ``M'' by chick heart muscle cells in culture (Cahn, 1969), but oxygen tension has its isozyme inducing effect regardless of the presence of metabolic inhibitors, including fluoroacetate, potassium cyanide, 2,4-dinitrophenol, and sodium amytal (Johansson, 1966). Because of this, it has been speculated that oxygen might act directly on the genes regulating LDH isozyme synthesis, but other methods of control, such as preferential degredation (Pruitt, 1971; Greengard, 1967) might be more fruitful to investigate experimentally, especially considering the greater lability of the ``liver'' isozyme, both in heat and cold (Zondag, 1963; Kreutzer and Fennis, 1964). Direct action of oxygen or peroxides on the labile isozyme, or an effect involving sequestering of the stabilizing coenzyme, NAD, might be mechanisms for regulating isozyme proportions. The H$_{4}$ isozyme seems to be the most lipophillic as well as the most stable, and the M$_{4}$ the least lipophillic (Kabara and Konvich, 1972). This result is implicitly predicted by the theory which relates cytoplasmic phase changes to control of enzyme activity (Peat and Soderwall, 1972).

Finding the H$_{4}$ isozyme increased with age leads to at least two or three questions:

\begin{center}
\begin{adjustwidth}{0.1\linewidth}{0.1\linewidth}
\begin{itemize}[label={}, leftmargin=*, rightmargin=0pt]
    \item 1)~Does enzyme induction by oxygen relate to the rate of consumption of oxygen, rather than to oxygen tension itself?
    \item 2)~If the function of this isozyme is to save some pyruvate for another use, rather than converting it to lactate (Lehninger, 1970), is carboxylation to oxaloacetate that use?
    \item 3)~Would it lead to a situation in which (because of H-LDH's inhibition by high concentrations of pyruvate) pyruvate carboxylase tended to become rate limiting for glycolysis?
\end{itemize}
\end{adjustwidth}
\end{center}

Although no mechanism is suggested, it is interesting that the age shift of isozymes is opposite to the shift which occurs in pregnancy, as early as day 5, when uterine oxygen consumption is known to decrease (Battelino, et al., 1971).

$\beta$-Glucuronidase is increaed in liver, uterus, kidney, and other tissues by injection of estrogen (Fishman, 1950) or various poisons, and this effect has been likened to the induction of detoxifying enzymes in the liver by their substrates. Levvy (Karunairatnam and Levvy, 1949) has offered the contrary interpretation that $\beta$-glucuronidase is merely involved in tissue proliferation, regardless of what causes the proliferation. Prolonged estrogen treatment causes an eventual decline in $\beta$-glucuronidase, as it does of the mixed function oxidases (Song and Kappas, 1968). Its decline with age in the hamster uterus thus is consistent with the view that estrogen tends to dominate uterine metabolism in senescence, but of course doesn't imply that this is what occurs. However, if its normal function in the uterus and other tissues is to conjugate and remove estrogen to prevent estrogen toxicity, then its absence in the senescent uterus would suggest an explanation for other physiological changes that occur in the uterus with age. If estrogen no longer turned over in the uterus, then a much smaller supply could raise the uterine content, Estrogen treatment is known to ``induce'' large amounts of the estrogen binding protein (Kraay and Black, 1970), as well as reducing the progesterone binding capacity of the uterus (Armstrong and King, 1971). Post-menopausal women have at least as high a concentration of uterine ``estrogen receptor'' protein as do young women (McGuire, et al., 1972; Trams, et al., 1971).

$\beta$-Glucuronidase is known to have an \textit{in vitro} transferase function (e.g., transfer of glucuronic acid from phenolphthalein to an alcohol) besides it common hydrolytic function (Fishman and Green, 1957). It has been shown to have a slow synthetic function \textit{in vitro} (Florkin, 1940). Fishman (personal communication) feels that an \textit{in vivo} conjugating (transferase) function is plausable, considering the similar role of a ``hydrollytic'' enzyme such as glucose-6-phosphatase. $\beta$-Glucuronidase is so abundant in smooth endoplasmic reticulum that it is considered likely to be a structural protein (Wakabayashi, 1970).

CATALASE: Catalase activity of the uterus is interesting because it is believed to relate to the cytochrome system (Deisseroth and Dounce, 1970), is lower in the female, at least in some species (Adams, 1965), ans since either its rate of synthesis or rate of degradation appears to change with age in mouse liver (Baird and Samis, 1971). The fact that its activity is lowered by riboflavin deficiency (Adams, 1965), and by serum or tissue extract from animals with tumors (Nakahara and Fukuoka, 1959), and is lower in females, suggests that its level will correspond to the amount of energy derived form mitochondrial oxidation. A similar extract is an abortifacient. The lower level in the aged uterus would not be particularly interesting if oxygen consumption were lower in that tissue. Since oxygen consumption tends to be higher, low catalase activity suggests that the oxidation may not be occurring in the usual way.

PEROXIDASE: The fact that estrogen activates the NADH-oxidase function of peroxidase, which itself is ``induced'' by estrogen (Lucas, et al., 1955; Yokota and Yamazaki, 1965; Jellink and Lyttle, 1971; Temple, et al., 1959; Hollander and Stephens, 1959; Beard and Hollander, 1962) suggests that in the uterus peroxidase may be involved at an early stage in the process of cell activation by estrogen. Vitamin E, which is known to inhibit peroxidation, has anti-estrogenic effects, including inhibition of the formation of the estrogen-dependent uterine age pigment, which is associated with NADH-oxidase activity, and which is generally considered to result from peroxidation of lipids and proteins (Elftman, et al., 1949).

VITAMIN E: Two classical symptoms of a vitamin E deficiency are increased tissue oxygen consumption (Houchin, 1942) and anemia (Fitch, 1968). The degree of stability of erythrocytes in the presence of oxidants, such as hydrogen peroxide, is an indication of the vitamin E status of the animal. Vitamin E has an anti-oxidant function, but other properties are also believed to be involved in its vitamin activity. It increases the viscosity of protein solutions (e.g., albumin), apparently by causing a more extended conformation of the protein. Such a physical or mechanical property, as well as a chemical anti-oxidant property, could be involved in its anti-hemolytic effect, and in its ability to stabilize other ``membrane'' systems, such as lysosomes (McCay, et al., 1971). It lowers the extractability of muscle proteins (Matusis, 1970). The enzymes of vitamin C synthesis in microsomes are inactivated by lipid peroxides, and preserved by added $\alpha$-tocopherol (Chatterjee and McKee, 1965). Estrogen's effects on vitamin C synthesis, $\beta$-glucuronidase (of lysosomes), and pigment formation may be produced by a common structural event. uterine age pigment requires estrogen for its development, but vitamin E treatment prevents its formation (Kaunitz, et al., 1948; Atkinson, et al., 1949). If hemes, or other porphyrins, are related to the lipofuscins or age pigments, they may be responsible for the NADH-oxidase activity that Bjorkerud (Strehler, 1967) found associated with the pigments. The same hemolysis which lends to vitamin E deficiency anemia could therefore also be responsible for the increased oxidation of vitamin E deficiency. Since hypoxia stimulates porphyrin synthesis (Falk, et al., 1959), oxygen wastage by porphyrins would tend to be a positive feedback process. The oxidase activity of age pigment granules is cold-inactivated (Bjorkerud and Cummins, 1963), which is consistent with the theory of pathway regulation by ``structural temperature'' (Peat and Soderwall, 1972). If estrogen's action is intrinsically realted to a hypoxic state, this could also explain the fact that porphyrins are synergistic with estrogen, and similarly could explain the occurrence of porphyrins mainly in female and cancer-prone rodents (Strong, 1942; Figge, 1942). Vitamin E is known to lower serum bilirubin related to liver damage (Schwarz, 1949).

The circulatory availability of oxygen, as well as the metabolic disposition of oxygen, may be directly related to the vitamin E status of the animal. Adaptive hyperemia in several glands has been found to result from the action of peptide kinins, produced by proteolytic kallikreins (Schachter, 1969). Small molecule inhibitors (e.g., Trasylol, $\varepsilon$-amino caproic acid) are known, as well as the protein inhibitors, of the kallikreins and other proteolytic enymes (cathepsins, trypsin). Age, x-irradiation, and estrogen have been implicated in various ways in proteinase inhibition. The ``Schute test'' (Schute, 1937; de Oliveira, 1949), a measure of the level of serum proteolytic inhibitory activity, clearly shows that a vitamin E deficiency increases the inhibition of proteolytic enzymes. The presence of inhibitors would tend to suppress formation of kinins, the local ``inflammatory'' vasodilators, and would also interfere with clot removal.

The hamsters that were given vitamin E in very large doses showed chronic leucocytic vaginal smears, as in rats given daily doses of progesterone (Labhsetwar, 1970).

A vitamin E deficiency causes muscles of rabbits (Morgulis and Oscheroff, 1937) to increase their concentration of ``extracellular'' ions, especially Na$^{+}$, Cl$^{-}$, and Ca$^{++}$, suggesting a possible loss of organic ions, as well as an energy loss: creatine phosphate may be largely responsible, since vitamin E deficiency causes creatinuria.

The observations of Thorneycroft and Soderwall (1969) regarding fewer and smaller corpora lutea in senescent hamsters may relate to the observations of Lecoq and Insidor (1949) that vitamin E deficient rats have no corpora lutea.

Progesterone's effect on Na$^{+}$, Cl$^{-}$, and Ca$^{++}$, and water is similar to that of vitamin E, and opposed to estrogen's effect. The most sensitive demonstration of vitamin E's anti-estrogen effect is probably Leqoc's (1949) on uterine chronaxy.

Vitamin E requirements increase with age (Fuhr, et al., 1949; Ames and Ludwig, 1964; P'an, et al., 1949; Emerson and Evans, 1940) and a deficiency acts mostly to block implantation (Blandau, et al., 1949).

\section{Oxygen Consumption}

A likely source of error in measuring tissue oxygen consumption (besides the problem of balancing cell damage in thin slices or minces against limited diffusion in thicker slices: Jusiak, 1970) is in neglecting for other gaseous metabolites (Dudka, et al., 1971; Maharajh and Walkley, 1972) than oxygen and carbon dioxide, of which ammonia is likely to be the most important by far. Tashiro (1922) observed ammonia production in nerve and muscle, which increased when the tissue was stimulated. Needham (1942) wrote that ammonia is not likely to be produced by embryonic tissue unless sufficient carbohydrate is not available, implying that it results from protein catabolism. A popular contemporary argument (Lowenstein, 1972) is that ammonia emission results from the activity of adenylate deaminase, but in amny cases glutamic dehydrogenase seems a more likely enzyme. In estrogen stimulation protein synthesis and turnover are accelerated (Mueller, 1968); glutamic acid is an important link between carbohydrate metabolism and proteins, and is also closely related to synthesis of pyrroles. Aspartate, however, seems to be the first amino acid synthesized after estrogenic stimulation (Jervell, et al., 1958), probably by amination of oxaloacetate (Barker and Warren, 1966). Glutamate dehydrogenase may have alanine dehydrogenase activity when dissociated into subunits. Estrogen can regulate this dissociation and change of activity, probably by increasing the supply of NADH (Tomkins, et al., 1961). Estrogen-induced L-alanine pyruvate conversion by this form of glutamate dehydrogenase would account for some of the diversion of pyruvate into protein synthesis. Lacking carbohydrate, a reversal of the reaction could form pyruvate, with a high rate of protein catabolism and production of ammonia. (A shift of the enzyme activity away from glutamate dehydrogenase may spare glutamate for synthesis of $\delta$-levulinic acid, and porphyrins.)

Senescent uterus has reduced glycogen levels in pregnancy (Connors, 1969), and if this lack of glycogen exists in the cycling senescent uterus, or if the rate of metabolism is high in relation to available carbohydrate, aged uterus may release more ammonia than young uterus. \textit{In vitro}, where small glycogen deposits may be quickly consumed and no external glucose is provided, the tissue with the highest metabolic rate will probably release the most ammonia, and if this is not taken into account, the manometric method will read this a diminished oxygen consumption. In the first measurements, samples often showed negative ``oxygen consumption,'' and this occurred more often with senescent tissue, especially myometrium, and probably was intensified by delay in preparing the tissue. Although the literature on uterine oxygen consumption does not mention this problem, it may be a general factor in measurements of oxygen consumption. When sulfuric acid was added to the vessel side arms, oxygen consumption measurements became consistently positive and rates became more consistent, especially in the case of senescent tissue.

Dissolved oxygen measurements would tend to show alteration in the opposite direction when ammonia is released, so additional measurements were made in the Beckman apparatus. No removal of ammonia is possible in this case, so the data should be considered only in relation to the manometric data, unless it can be assumed that the much shorter time needed to get measurement eliminates the factor if ammonia production.

Increased oxygen consumption by estrogen-activated tissue does not imply increased mitochondrial oxidation, coupled to phosphorylation or not. Lucas, et al., (1955) determined that what had been assumed to be elevated cytochrome oxidase activity was actually the NADH-oxidase activity of peroxidase. Wilson (1969) reported the puzzling observation that estrogen increased the activity of only malate dehydrogenase of all the citric acid cycle enzymes, and that others may have decreased. It is interesting that MDH is one of the few enzymes known to increase in some tissues in senescence (Finch, 1972). The carboxylation of pyruvate to oxaloacetate, followed by amination to form aspartic acid (Barker and Warren, 1966; Jervell, et al., 1958) may have its equivalent in this situation, viz., production of oxaloacetate for amino acid formation.

Barker, Neilson, and Warren (1966) claim that lipid synthesis, mediated by NADP$^{+}$, regulates the hexose monophosphate shunt. (Elsewhere it has been suggested how a cytoplasmic phase change could accelerate lipid synthesis: Peat and Soderwall, 1972). They found that the specific activity of glucose-6-phosphate dehydrogenase decreased significantly during an early period in which G6P and 6PG were being oxidized at an increased rate. Activation of either lipid synthesis or oxidation of NADPH would be able to stimulate this pathway.

\section{Oxygen Tension}

The measurements of intrauterine pO$_{2}$ that have been made with the Beckman microelectrode inserted through an 18 gauge arterial needle (Yochim and Clark, 1971; Mitchell and Yochim, 1968 ,1969b; Salderini and Yochim, 1967) are apparently inaccurately high. The sharp end, and additional thickness, possibly cause stil-oxygenated blood to be free into the cavity surrounding the electrode tip, where it would not be subject to oxidation by the tissue. The cyclic variation of about 20 mm. Hg pO$_{2}$ could be a further artifact, since the electrode itself reduces oxygen and will quickly lower the oxygen tension of a small volume of fluid that is not stirred. The end of the needle immobilizes a few microliters of liquid. Periodic contractions of the uterus would replace the fluid in the tip of the needle with fresh blood (or another oxygenated fluid produced by introducing the electrode) from elsewhere in the lumen. This procedure gave measurements in the hamster's uterus that were about the same as those obtained by Yochim and others in the rat, except that instead of one minute cycles, in the hamster the cycles were about three to four minutes long. Different means or maxima determined by this method would therefore reflect indirectly the rate of oxygen consumption by measuring the level of oxygen present in the blood which is released into the lumen. The rate of oxygen consumption would not be entirely unrelated to tissue pO$_{2}$ (yochin, 1971), but it would present an entirely false picture of the conditions experiences by the pre-implantation embryo.

Using the ``bare'' electrode (with stainless steel jacket, but no needle around it) with a diameter of 1.0 mm. and a blunt tip, a much lower pO$_{2}$ was measured, and there was no cyclic change that could be recorded. The oxygen tension is so low that an accurate absolute measurement is extremely difficult, but relative measurements of differences between animals can be easily and clearly made. The oxygen tensions observed are lower than any known mammalian tissue except cancer. Senescent animals' uteri had a pO$_{2}$ in the range of 0-2 mm. Hg, young animals in the range of 4-8 mm. Hg, and vitamin E treated senescent, 6-12 mm. Hg. The color of the uterus varied from maroon in the old animals to pale pink in the vitamin E treated animals; young animals' uteri were bright red.

At such low oxygen tension it might be advantageous to have hemoglobin with high oxygen affinity; it has been reported that sheep with hemoglobin A (with higher than normal oxygen affinity) are better able to withstand the abortifacient effects of estrogenic pasture than are other sheep (Obst and Seamark, 1971). A lower ``unloading tension'' would correspond to a lower tissue oxygen tension, with possibly adaptive effects, such as a more controlled or gradual release of oxygen. Oxygen consumption decreases at implantation sites in the rat uterus at the time of implantation (Suranu and Heald, 1971), and this probably causes a rise in pO$_{2}$ to levels the embryo requires. Progesterone by counteracting estrogen, may be the cause (Goodland, et al., 1953). As early as the fourth day of pregnancy there are greatly increased obstructions in the uterus, probably from increased adhesiveness of the epithelium which at this time is closing over the blastocysts. Even the ``bare'' electrode has to be forced if it travels more than a centimeter from the opening, and obviously is causing luminal bleeding. The force required was such that at one obstruction the electrode passed entirely through the wall of the uterus. before hitting an obstruction the pO$_{2}$ reading was 40 mm. Hg, which might have been high because of being near the opening, though the soft and abundant endometrium formed a seal. Beyond the obstructions the pO$_{2}$ measured 5-10 mm. Hg, or about that of the vitamin E treated cycling hamsters. The average blood would be expected to have more oxygen than the respiring tissue, but if the deeper stromal or myometrial tissue were respiring faster than the implanting epithelium, bleeding might cause a lowering of pO$_{2}$.

Yochim's system of measuring uterine luminal oxygen tension with an aterial needle may be measuring something equivalent to the ``bulk'' pO$_{2}$ in organ culture, i.e., possibly uterine fluid that has been contaminated with oxygenated blood or gas introduced with the needle, so they are possibly correct about the effect of pO$_{2}$ on LDH isozyme proportions in the uterus ,which they compare to the effects in cultured uteri.

However, the pO$_{2}$ at the surface of the cell in culture, where the effect of pO$_{2}$ on LDH isozymes is to be studied, will be much lower, becaus eof an O$_{2}$ gradient in the unstirred layer. A bare electrode, able to contact the epithelial tissue in the uterus, and less likely to cause bleeding, would probably more closely approach the pO$_{2}$ actually experienced by tissue either \textit{in vivo} or in culture.

The pO$_{2}$ Yochim's group measured would not, according to Needham's review (1931), be low enough to approach limitation of embryonic development. However, when the lumen ``surface'' pO$_{2}$ is measured, by the bare electrode, the tension is seen to be lose to that which limits respiration and development of embryos; the actual pO$_{2}$ the embryo experienced in the experiments reviewed by Needham is not known, nor is the actual pO$_{2}$ in the old hamster's uterus, since it is at the extreme lower limit of the measuring system's capacity. A better technique is needed, to determine the real minimum.

\section{TTC Reduction}

The reduction of triphenyl tetrazolium chloride is considered to reflect dehydrogenase activity (Racker, 1955; Novikoff, 1959) or the cell's reductive capacity, including substrate availability, and is believed to be one of the most sensitive indicators of estrogenic stimulation. It is known to accurately correspond to QO$_{2}$ in many tissues (Cascarano and Zweifach, 1955). Martin (1959) has reported that the TTC reducing response of the ovariectomized mouse vagina to exogenous estrogen is linear in relation to dosage from about two picograms of estradiol per mouse up to about one microgram. Estrogen increased phosphorylase activity (Takeuchi, et al., 1962) and ATPase activity, so would tend to release inhibition of glycotic enzymes while providing increased amounts of glucose. Ammonia is a very effective activator if phosphofrutokinase, and may be produced by cell activation (Lowenstein, 1972). The production of NADH would therefore be increased. The earliest effect of estrogen may be on glycolysis, rather than on respiration (Roberts and Szego, 1953). Since estrogen causes ``induction'' of peroxidase, which has NADH oxidase activity (for which estrogen serves as a coenzyme: Beard and Hollander 1962), at least part of the increased NADH would tend to reduce oxygen by this short-circuit route, increasing oxygen consumption without necessarily increasing usable oxidative energy production. That is, an extramitochondrial uncoupling of oxidative phosphorylation could occur though these known ezymic effects of estrogen. Whatever the mechanism, increased oxygen consumption must lower the pO$_{2}$ of a tissue of circulation doesn't increase proportionally. According to Yochin, et al., (1971) circulation doesn't increase as fast as the mas of tissue increases, under estrogenic stimulation. With the rapid ``uptake'' of water (or possibly metabolic production of water) immediately after estrogen treatment, the average distance of cytoplasm from capillaries must increase, which would possibly decrease the efficiency of tissue oxygenation.

The fact that the senescent hamster uterus shows a distinctly increased TTC reduction is probably the best evidence that senescent sterility is the result of an active physiological process (or malfunction). Schultz (1964, 1965, 1966, 1967, 1968) has found that strain differences in rats suggestive of estrogen differences correlate with fertility and TTC reduction. Selection for large bodies produced low uterine TTC reduction and large litters, while selection for small bodies produced high uterine TTC reduction, and small litters. He also found increased TTC reduction at the time of senescent litter loss by resorption in rats.

Racker's (1955) ``nothing dehydrogenase'' of well-dialyzed protein fraction, which give a string nitroprusside reaction, reduces NAD or NADP without the addition of substrate, and the reduced nucleotide can then reduce tetrazolium. Racker and others have suggested that protein SH groups are the source of this reducing capacity. The present tendency is probably to assume that starches or fats contaminate the protein fraction and provide the reductive capacity. However, G. Ungar (1957, 1959) has presented some evidence to support the view that cell excitation involves an increase of protein SH groups, resulting from conformation changes or ``partial denaturization'' after excitation or hormone treatment. The technique of determining uterine protein SH concentration is very hard to use reliably, but might eventually be useful to study the more exact nature of the uterine reducing capacity which changes with age and estrogen treatment.

\section{Age Pigment}

Uterine ``age pigment,'' which develops in the presence of estrogen and vitamin E deficiency (Kaunutz, et al., 1949), is not noticeably soluble in ethanol, benzene, benzyl benzoate, acetone, or chloroform, but is highly soluble in alkaline pyridine solution, which is an excellent bile pigment solvent, and is partly soluble in ether, which is also a solvent for certain bile pigments (With, 1968), and the spectrum is somewhat similar to that of a blood extract. The presence of an ether soluble fraction of bile pigment of a certain magnitude in human serum is considered to be indicative of cancer (With, 1968). Bjorkreud (1964) has observed that age pigment has NADH oxidase activity, and contains the same heme group. Aged uteri (Zinkat, 1931) contain remarkably large amounts of iron, which might be bound by the hemes. The fact that the pigments appear under conditions in which red blood cells are hemolyzing rapidly, further supports the possibility that they are porphyrin-related (Gyorgy and Rose, 1949). Bjorkerud (1964) also observed that this NADH oxidase activity of the pigment is cold-inactivated, which tends to support the theory of entropic activation of oxygen consumption (Peat and Soderwall, 1972).

\section{NMR}

The evidence is very good that water near interfaces is structured differently from bulk water (Drost-Hansen, 1971, 1972). Probably the most dramatic demonstration was Shereshevsky's (1928) experiment showing very low vapor pressure of the waster in capillary tubes. What is still in dispute is whether the particular structure induced by a surface represents an intrinsic ``phase'' or crystal-like state of water (Drost-Hansen, 1971, 1972), or whether the particular repeating pattern of the surface (such as a macromolecule) imposes its structure onto the water (Ling, 1972). It has been pointed out that the enzyme biochemists (the ``Pauling School'') emphasize the role of ionic side chains of proteins in ordering (or binding) water, while the fiber chemists emphasize the role of the protein backbone (Lin, 1972). Since estrogen stimulation involves a decrease in protein, DNA, and RNA concentration, because of the great increase in water concentration, the average distance of intra-cellular water from macromolecular surfaces must increase, and this should mean that the water molecules experience fewer restrictions of orientation. High resolution NMR spectra show a distinct broadening of the water line in tissue, relative to bulk water, and less broadening (with a shift upfield) in estrogen-treated uterus than in atrophic uterus from ovariectomized hamsters. Old uteri were not significantly different from young uteri, and both were intermediate between tissue from ovariectomized and estrogen-treated animals. The method, of course, can't determine the quantity of extracellular and intracellular water, unless their movement or shielding is very different, and this apparently is not the case. Diffusion, proton exchange, paramagnetic ions, and lipids are all probable sources of error in this method.

Spin echo NMR can eliminate the possibility that the line broadening results from proton exchange, and shows a slight difference between old and young uteri ,as well as between estrogen-treated and untreated young animals, with young uteri showing the greatest difference from bulk water. The old uteri, by being intermediate between younger uteri and bulk water, are somewhat analogous to the malignant tumors in Damadian's study (1971). Even very dense connective tissue (rat knee joint) responds to irritation by a similar change in water relaxation time, without a large uptake of water (Rokityanskiy and Yakimov, 1971).

\section{Harderian Fluorescence}

This strain of hamsters apparently has less active porphyrin secretion from the Harderian glands than do many other rodents, unless their diet is simply richer in B vitamins. A B vitamin (especially pantothenic acid) deficiency is known to cause increased porphyrin secretion by the Harderian glands. Seince World War II, animal feeds have become more standardized and are generally supplemented with adequate vitamins, and this may account for the loss of interest in the Harderian glands and their porphyrin secretions. Most (about 80\%) barely senescent (14 months) hamsters that have been examined in this la showed no fluorescence of the eye socket, or only a weak green fluorescence. Beyond this age, however, the typical bright red fluorescence becomes more common. By the time their body weight begins to decrease, which is usually after 15 months, most of the females showed some red fluorescence, but no male did. This is consistent with the observations of Strong (1942) and Figge (1942) regarding the relation of the secretion to estrogen (no males showed fluorescence) and possibly to the likelihood of an animal to develop cancer. Its appearance around the time of reproductive senescence suggests that the liver function may be decreasing at this time, and that the pigment may be related to the altered uterine metabolism.

Porphyrins are synergistic with estrogen, as well as with carcinogens (Strong, 1942; Figge, et al., 1942; Bittner and Watson, 1946, Kennedy, 1970), and the pyrroles and porphrins are also conjugated by the liver for excretion via the kidneys. With liver damage, they accumulate, causing porphyria (With, 1968). at least in rodents, concentration of porphyrins in the Harderian glands corresponds to estrogen levels and to the tendency to develop cancer, as well as to deficiencies of the B vitamins essential for liver conjugation (Figge, 1944; Hoffman, 1972).

\section{Uterine Weight, Water Content, and Thickness}

Organ weight has been said to be a more sensitive index to metabolic change than any other single condition (Criscuolo, et al., 1955; Rusin, 1971). All dimensions of the hamster uterus that were observed, diameter, thickness of endometrium and myometrium, and total weight, increased with age, but there was no significant difference between young and old endometria's water content.

Though all parts of the old uterus are enlarged, the biggest change with age appears to be the thickness of the myometrium. A large part of the collagen that accumulates with age is in the myometrium (Loeb, 1939).

Estrogen inhibits collagenase (Jeffery, et al., 1971; Woessner, 1968) and increases collagen synthesis (Henneman, 1968; Kao, et al., 1964). Loeb (1939) found that estrogen and age both cause collagen accumulation in the uterus. Schaub (1965) found a three-fold increase of uterine collagen in the aged rat, and observed the same age changes (crosslinking) that occur in other tissues.

Finn (1963) found that collagen accumulates in mouse uterus with age, both in a barren uterine horn and in a repeatedly pregnant horn, when the mouse was unilaterally ovariectomized. This weakens the arguments that maintain that collagen formation is largely the result of mechanical stress.

Alkaline phosphatase activity increases with estrogenic stimulation. Therefore, if excessive estrogenic stimulation is responsible for the failure of implantation in senescent animals because of an inadequate level of progesterone, alkaline phosphatase activity should be demonstrable.

In slides prepared by T. J. Connors for a study of enzyme changes in implantation (1969), at all times examined--days 3$\frac{1}{2}$, 4, 4$\frac{1}{2}$--the epithelium of the senescent uterus was more intensely stained. The senescent stroma was more intensely stained on day 4$\frac{1}{2}$, but on day four it was in several cases more lightly stained than in the young.

Since the decidual cell reaction (to an artificial stimulus, viz., the injection of air: Stockton, 1972) is delayed by about twelve hours in the old hamster (and also fails to produce a weight increase as great as in the young animal), this difference in stromal alkaline phosphatase probably corresponds to that delay. Horikoshi and Weist (1971), among others, have attributed ``uterine refractoriness'' to a small progesterone/estrogen ratio.

The epithelial reaction, however, seems to reflect a chronic difference. The senescent myometrium stains nearly as much as the endometrial stroma, but the young myometrium is almost unstained, on all days examined. Again, this suggests a possible chronic difference in degree of ``estrogenic'' stimulation.

\section{An Organismic Change}

These diverse but interlocked observations suggest that a profound reorganization, probably consisting of many compensatory tissue responses to the altered internal environment, occurs during aging. It seems that these changes may be most easily observable in the female mammal, but analogous processes may be practically universal. The mutual adjustment of many parts of a system makes the evaluation of a single component---such as estrogen---very difficult and ambiguous, yet the ``meaning'' of the configuration to the organism is reasonably consistent and determinate. If the many relevant components of the system can be identified, then it should be possible to evaluate each part without ambiguity. A partial example is the estrogen/progesterone ratio, but it will probably eventually be possible to interpret quantitatively the concentration of other hormones (especially the salt and sugar regulators), the density and composition of connective tissue, composition of bile pigments, enzymes, etc. For an organismic process, such as aging, it will be appropriate to have precise and quantifiable organismic concepts.