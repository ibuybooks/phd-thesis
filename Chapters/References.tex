\chapter{References}

% 1
Adams, C. E. (1970) Ageing and reproduction in the female mammal with particular reference to the rabbit. J. Reprod. Fertil. 12: 9-16.

Adams, D. H. (1955) Liver catalase in the riboflavin-deficient mouse. Biochem. J. 60: 568.

Alexander, P. and D. I. Connell (1963) The failure of the potent mutagenic chemical, ethyl methane sulphonate, to shorten the life-span of mice. In: ``Cellular Basis and Aetiology of Late Somatic Effects of Ionizing Radiation'' (R.J.C. Harris, ed.), p. 259. Academic Press, N.Y.

Allfrey, V. G., R. Meudt, J. W. Hopkins, and A. E. Mirsky (1964) Sodium-dependent `transport' reactions in the cell nucleus and their role in protein and nucleic acid synthesis. Proc. Nat. Acad. Sci. 47: 907.

Angelova-Gateva, P. (1969) Tissue respiration and glycolysis in quadriceps femoris and heart of rats of different ages during hypodynamia. Exp. Geront. 4: 117.

Armstrong, D. T. and E. R. King (1971) Uterine progesterone metabolism and progestational response: effects of estrogens and prolactin. Endocrinol. 89: 191.

Arvay, A., I. Takacs, P. Ladanyi, A. Balough, and K. Benko (1971) The effect of intensive nervous stimulation on certain physico-chemical properties of rat tail tendon and uterus colagen. Gerontologica 17: 157-169.

Atkinson, D. E. (1971) Limitation of metabolite concentrations and the conservation of solvent capacity in the living cell. Ann. N.Y. Acad. Sci. 52(3): 68-71.

Atkinson, W. B., H. Kaunitz, and C. A. Slanetz (1949) Ovarian hormones and uterine pigmentation in vitamin E deficiency. Fed. Proc. 8: 349.

Auletta, F. J. (1971) effect of epinephrine on implantation and foetal survival in the rabbit. J. Reprod. Fert. 27: 281-282.

% 2
Auberback, R. (1970) Presentation at Immunology Symposium, University of Oregon, November, 1970.

Baird, M. B. and H. V. S. Amis (1971) Regulation of catalase activity in mice of different ages. Gerontologica 17: 105-115.

Baird, S. L. Jr., G. Karreman, H. Mueller, A. Szent-Györgyi (1957) Ionic semipermeability as a bulk property. Proc. Nat. Acad. Sci. 43: 705-708.

Baker, S. P., N. W. Shock and A. H. Norris (1962) Influence of age and obesity in women on basal oxygen consumption expressed in terms of total body water and intracellular water. In: ``Biological Aspects of Aging,'' (N. W. Shock, ed.), p. 84., Colombia University Press, New York.

Barker, K. L., M. H. Neilson, J. C. Warren (1966) Estrogen control of uterine carbohydrate metabolism: Factors affecting glucose utilization. Endocrinol. 79: 1069.

Barker, K. L. and J. C. Warren (1966) Estrogen control of carbohydrate metabolism in the rat uterus: Pathways of glucose metabolism. Endorinol. 78: 1205-1212.

Barnett, S. A. and K. M. H. Munro (1971) Persistent corpora lutea of mice in a cold environment. Nature 232: 406-407.

Battellino, L. J., J. Sabulsky, and A. Blanco (1971) Lactate dehydrogenase iso-enzymes in rat uterus: Changes during pregnancy. J. Reprod. Fert. 25: 393-399.

Beard, J. and P. Hollander (1962) The nature of uterine phenol-activated oxidation of DPNH. Arch. Biochem. Biophys. 96: 592-600.

Bell, E. (1971) Report on First International Conference on Cell Differentiation, sections I and IV: Embryology and Cell Biology, Dev, Biol, 26(3): 505.

Biggers, J. D. (1969) Problems concerning the uterine causes of embryonic death, with special reference to the effects of aging on the uterus. J. Reprod. Fert. Suppl, 8: 27-43.

Biggers, J. D. (1971) New observations on the nutrition of the mammalian oocyte and the preimplantation embryo. In: ``Biology of the Blastocyst'', (R. J. Blandau, ed.), p. 319, University of Chicago Press, Chicago.

% 3

Biskind, M. S. (1946) Nutritional therapy of endocrine disturbances. Vitamins and Hormones 4: 147-170.

Bjorkerud, S. and J. T. Cummins (1963) Selected enzymic studies of lipofuscin granules isolated from bovine cardiac muscle. Exp. Cell Res. 32: 510.

Bjorkerud, S. (1964) Isolated lipofuscin granules. A survey of a new field. In: ``Advances in Gerontological Research,'' Vol. I, (B. L. Strehler, ed.), p. 257. Acad. Press, N. Y.

Blaha, G. C. (1971) Ovarian steroid dehydrogenase and plasma progesterone levels in aged golden hamsters. Anat. Rec. 169: 279 (abstr).

Blaha, G. C. (1964) Reproductive senescence in the golden hamster. Anat. Rec. 150: 405-412.

Blandau, R. J., H. Kaunitz, C. A. Slanets (1949) Ovulation, fertilization, and transport of ova in old, vitamin E deficient rats. J. Nut. 38: 97.

Bloch, S. (1961) Untersuchungen uber Klimakterium und Menopause on Albino-Ratten. III Mitleitung. Gynaecologia 152: 414.

Bo, W. J., W. L. Poteat, W. A. Krueger, and F. McAlister (1971)
The effect of progesterone on estradiol-17$\beta$-dipropionate-induced wet weight, percent water, and glycogen of the rat uterus. Life Science 6.

Boag, J. W. (1970) Cell respiration as a function of oxygen tension. Int. J. Radiat. Biol. 18(5): 474-478.

Boling, J. L., R. J . Blandau, J. F. Wilson, W. C. Young (1939) Post-parturitional heat responses of new-born and adult guinea pigs. Data on partutition. Proc. Soc. Exp. Biol. Med. 42: 128.

Boulos, B. M., M. MacDougall, D. W. Shoeman, D. L. Azarnoff (1972) Evidence that inhibition of hepatic drug oxidation by tumors is mediated by a circulating humor. Proc. Soc. Exp. Biol. Med. 139(4): 1353-1355.

Brent, R. L. (1964) Modification of the teratogenic and lethal effects of irradiation to the mammalian fetus. In: ``Effects of Ionizing Radiation on the Reproductive System'' (W. D. Carlson and F. X. Gassner, eds.), MacMillan, N.Y.

% 4

Brinster, R. L. (1971) Mammalian embryonic metabolism. In: ``Biology of the Blastocyst'' (R. J. Blandau, ed.), University of Chicago Press, Chicago.

Brooksby, G., F. Sahinin and A, L. Soderwall (1964) X-Irradiation effects on reproductive function in the female hamster. In: ``Effects of Ionizing Radiation on the Reproductive System'' (W. D. Carlson and F. X. Gassner, eds.), MacMillan, New York.

Brown, S. W., G. M. Krise, H. B. Pace and J. DeBoer (1964) Effect of continuous radiation on reproductive capacity. In: ``Effects of Ionizing Radiation on the Reproductive System'' (W. D. Carlson and F. X. Gassner, eds.), p. 103, MacMillan, N. Y.

Bruce, N. W. (1972) Resistance of the rat embryo to ligation of a uterine artery early in gestation. J. Reprod. Fert. 28: 265-267.

Bullough, W. S. (1971) Ageing of mammals. Nature 229: 608-610.

Burack, E., J, M. Wolfe, W. Lansing and A. W. Wright (1941) The effect of age upon the connective tissue of the uterus, cervix, and vagina of the rat. Cancer Res. 1: 227.

Butturini, V. (1949) Clinical and experimental studies in vitamin E. Ann. N.Y. Acad. Sci. 52(3): 397-405.

Bykhovtsova, T. L. (1970) The effect of ginseng and eleutherococcus root preparations on carbohydrate metabolism. Isvestiya Akademia Nauk SSSR, Seriya Biologicheskaya, No 6: 916.

Cahn, D. (1964) Developmental changes in embryonic enzyme patterns: The effect of oxidative substrates on lactic dehydrogenase in beating chick embryonic heart cell cultures. Devel. Biol. 9: 327-346.

Calloway, N. O. and B. S. Kujak (1966) Patterns of water, ash and organic-matter changes in senescence, J. Am. Ger. Soc. 14: 505.

Carpenter, D. G. and J. A. Loynd (1968) An integrated theory of aging. J. Amer. Ger. Soc, 16(12).

Casarett, G. W. (1963) Concept and criteria of radiologic ageing. In: ``Cellular Basis and Aetiology of Late Somatic Effects of Ionizing Radiation'' (R. J. C. Harris, ed.), p. 189, Academic Press, N. Y,

Cascarano, J. and B, W. Zweifach (1955) Comparative histochemical and quantitative study of adrenal and kidney tissue by tetrazolium technique. J. Histochem. Cytochem. 3: 369.

% 5
Challis, J. R. G., R. B. Heap, and D. V. Illingworth (1971) Concentrations of oestrogen and progesterone in the plasma of non-pregnant and pregnant and lactating guinea-pigs. J. Endocr. 51: 333-345.

Chatterjee, I. B. and R. W. McKee (1965) Biosynthesis of L-ascorbic acid in rat liver microsomes: Influnces of age, sex, dietary changes, and whole-body x-irradiation. Arch. Biochem. Biophys. 109: 62.

Chatterjee, S. N., V. P. Kamboj, and A. B. Kar (1971) Effect of intrauterine contraceptive suture on corpura lutea of guinea pigs. Indian J. Exp. Biol. 9: 105.

Chvapil, M., J. Hurych, E, Mirejovska (1970) Effect of long term hypoxia on protein synthesis in granuloma and in some organs in rats. Proc. Soc. Exp. Biol. Med. 135: 613-617.

Clark, H. and E. Gurpide (1972) Measurement of the solubility of progesterone under competitive protein binding assay conditions. J. Clin. Endocr. Metab. 34: 1085.

Clark, S. W. and J. M. Yochim (1971) Lactic dehydrogenase in the rat uterus during progestation, its relation to intrauterine oxygen tension and the regulation of glycolysis. Biol. of Reprod. 5: 152-160.

Claus, E. P., V. E. Tyler and L. R. Brady (1970) ``Pharmacognosy.'' Lea and Febiger, Philadelphia.

Clinch, J. and V. R. Tindall (1969) Br. Med. J. 5644: 602-605.

Comfort, A. (1964) ``Mechanisms of ageing, the biology of senescence.'' Holt, Rinehart, and Winston, N. Y.

Comfort, A. (1969) Geriatric Focus 8(17): 1.

Comolli, R. (1971) Hydrolase activity and intracellular pH in liver, heart and diaphragm of aging rats. Exper. Geront. 6(3): 219.

Cone, C. D., Jr. (1970) Variation of the transmembrane potential level as a basic mechanism of mitosis control. Oncology (Basel) 24(6): 438.

% 6

Connors, T. J. (1969) Reproductive senescence in the golden hamster: Early development and implantation of the blastocyst. Ph.D. Dissertation, University of Oregon, Eugene.

Connors, T. J., L. W. Thorpe, and A. L. Soderwall (1972) An analysis of preimplantation embryonic death in senescent holden hamsters. Biol. Reprod. 6: 131-135.

Cowdry, E. V. (1952) p. 672 In: ``Problems of Aging,'' 3rd Ed. (A. I. Lansing, ed.) Williams and Wilkins, Baltimore.

Crile, G, w. (1941) ``Intelligence, Power and Personality,'' McGraw-Hill, N.Y.

Csapo, A, (1950) Actomyosin formation by estrogen action. Amer. J. Physiol. 162: 406.

Damadian, R. (1971) Tumor initiation by nuclear magnetic resonance. Science 171: 1151.

Daniel, C. W., K. B. deOme, L. J. T. Young, P. B. Blair and L. J. Faulkin, Jr. (1968) The in vivo life span of normal and preneoplastic mouse mammary glands: A serial transplantation study. Proc. Natl. Acad. Sci. 61(1):53.

Daniel, C. W. and L. J. T. Young (1971) Influence of cell division on an aging process. Exp. Cell Res. 65: 27.

Daniel, J. C., Jr. (1964) Some effects of steroids on cleavage of rabbit eggs \textit{in vitro}. Endocrinol. 75: 706.

Daniel, C. W., L. J. T. Young, D. Medina, and K. B. deOme (1971) The influence of mammogenic hormones on serially transplanted mouse mammary gland. Exp. Geront. 6: 95.

Dawkins, R. (1971) Selective neurone death as a possible memory mechanism. Nature 229: 118.

Dawson, D. M., T. L. Goodfriend, N. O. Kaplan (1964) Lactic dehydrogenase: functions of the two types. Science 143: 929-933.

Devik, F. (1963) The effect of x-irradiation compared to an apparently sepcific early effect of skin carcinogens. In: ``Cellular Basis and Aetiology of Late Somatic Effects of Ionizing Radiation'' (R. J. C. Harris, ed.). Academic Press, N.Y.

De Waard, F. and J. H. H. Thijssen (1970) Relationship between hormonal cytology and steroid excretion in post-menopausal women. Endocrinol. 48(4): xxxvi-xxxvii.

% 7

Diczfalusy, E. G. Notter and R. Nissen-Meyer (1964) Influence of ovarian irradiation on urinary estrogens. In: ``Effects of Ionizing Radiation on the Reproductive System'' (W. D. Carlson and F. X. Gassner, eds.), p. 394. MacMillan, N.Y.

Dove, G. A., F. Morley, A. Batchelor, and S. F. Lunn (1971) Oestrogenic function in postmenopausal women. J. Reprod. Fert. 24: 1-8.

Dreisback, R. A. (1959) The effects of steroid hormones on pregnant rats. J. Endocrin. 18: 271.

Drost-Hansen, W. (1970) Structure and properties of water near biological interfaces. Lab. for Water Res., Coral Gables.

Drost-Hansen, W. (1972) Phase transitions in biological systems---manifestations of cooperative processes in vicinal water. Paper presented at New York Acad. of Sci. Internet. Conference on Interactions of Ions and Water with Macromolecules and Biological Systems, January, 1972. N.Y.

Drummond, J. C., R. L. Noble, and M. D. Wright (1939) Studies on the relationship of vitamin E (tocopherols) to the endocrine system. J. Endocrin. 1: 275.

Dubina, T. L. (1970) Effect of complexons on the aging process. Dokl. Akad. Nauk BSSR 14(9): 860.

Dudka, L. T., H. J. Inglis, R. E. Johnson, J. M. Pechinski, and S. Plowman (1971) Inequality of inspired and expired gaseous nitrogen in man. Nature 232: 265.

Dutra de Oliveira, J. (1949) The Shute test for checking unbalance produced by lack of vitamin E. Ann. N.Y. Acad. Sci. 52(3): 325.

Edgren, R. A., D. Peterson, M. A. Johnson, and G. C. Shipley (1961) Possible progesterone-blockage of estrogen-induced interruption of pregnancy in rats. Fert. and Ster. 12(2).

Eisenfeld, A. J, and J. Axelrod (1966) Effect of steroid hormones, ovariectomy, estrogen pretreatment, sex and immaturity on the distribution of H-estradiol. Endocrinol. 78: 38-42.

Elftran, H., H. Kaunitz, and C. A. Slanetz (1949) Histochemistry of uterine pigment in vitamin E deficient rats. Ann. N.Y. Acad. Sci. 52: 72.

Emerson, G, A. and H. M. Evans (1939) Restoration of fertility in successively older E-low female rats. J. Nutr. 18: 501.

% 8

Engel, L. L. (1970) Eli Lilly Lecture: Estrogen Metabolism and Action. Endocrin. 87(5): 827.

Ermini, M. (1970) Das Altern der Skelettmuskulatur. Gerontologia 16: 65-71.

Everitt, A. V. (1971) Food intake, growth and the aging of collagen in rat tail tendon. Gerontologia 17: 98-104.

Falk, J. E., R. J. Porra, A. Bown, F. Moss, and H. E. Larminie (1959) Effect of oxygen tension in haem and prophyrin biosynthesis. Nature 184: 1217.

Farley, T. M., J. Scholler, J. L. Smith, K. Folkers, C. D. Fitch. (1967) Biochem. Biophys. 121: 625-632.

Fernandes, G., E. J. Yunis, J. Smith, R. Good (1972) Dietary influence on breeding behavior, hemolytic anemia, and longevity in NZB mice. Proc. soc. for Exp. Biol. Med. 139(4): 265-274.

Field, E. J. (1976) The significance of astroglial hypertrophy in scrapie, kuru, multiple sclerosis and old age together with a note on the possible nature of the scrapie agent. Duetsche zeit. fur Nervenheilkunde 192: 265-274.

Figge, F. H. J., L. C. Strong, L. C. Strong, Jr., and A. Shanbrom (1942) Fluorescent porphyrins in Harderian glands and susceptibility to spontaneous mammary carcinoma in mice. Cancer Res. 2: 335.

Figge, H. J. (1945) The relationship of pyrrol compounds to carcinogenesis. Res. Conf. on Cancer (1944) p. 117-128.

Finn, C. A. (1966) The initiation of the decidual cell reaction in the uterus of the aged mouse. J. Reprod. Fert. 11: 423-428.

Finn, C. A. (1970) The aging uterus and its influence on reproductive capacity. J. Reprod. Fert. Suppl. 12: 31-38.

Finn, C. A. and C. C. Emmens (1961) The effect of dimethylstilboesterol and oestradiol on deciduoma formation in the rat. J. Repr. Fertil. 2: 528-529.

Finn, C. A., S. M. Fitch, and R. D. Harkness (1963) Collagen content of barren and previously pregnant uterine horns in old mice. J. Reprod. Fert. 6: 405-407.

% 9

Finn, C. A. and l. martin (1969) The cellular response of the uterus of the aged mouse to oestrogen and progesterone. J. Repord. Fert. 20: 545-547.

Fishman, W. B. (1950) $\beta$-glucuronidase. In: ``The Enzymes.'' (J. B. Sumner and K. Myrback, eds.), Vol. 1, Part 1, p. 635. Academic Press, N.Y.

Fishman, W. H. and S. Green (1957) Enzymic catalysts of glucuronyl transfer. J. Biol. Chem.

Fishman, W. H., B. Springer, and R. Brunetti (1948) Application of an improved glucuronidase assay meethod to the study of human blood $\beta$-glucuronidase. J. Biol. Chem. 173: 449-456.

Fitch, C. D. (1968) Experimental anemia in primates due to vitamin E deficiency. Vitamins and Hormones 26: 501-514.

Florkin, M., R. Crismer, G. Duchateau, and R. Houet (1942) Sur les $\beta$-glucuronides et la $\beta$-glucuronidase. Enzymologia 10: 220.

Folk, G. E., Jr., E. E. Grindeland (1962) Effects of cold exposure on the oestrous cycle of the golden hamster (\textit{Mesocricetus auratus}). J. Reprod. Fert. 4: 1-6.

Frampton, J. (1966) Increased oestrogen activity associated with postmenopausal bleeding. J. Obstet. Gyn. Br. Commonw. 73: 137.

Fredricsson, B. (1969) Histochemistry of the oviduct. In: ``The Mammalian oviduct'' (E. S. E. Hafez and R.J. Blandau, eds.), p. 311. University of Chicago Press, Chicago.

Friedman, S. M., C. L. Friedman, and M. Nakashima (1968) Adrenal-neurohypophyseal regulation of electrolytes and work performance: Age-related changes in the rat. In: ``Endocrines and Aging'' (L. Gitman, ed.), Thomas, Springfield.

Frolkis, V. V., V. V. Bezrukov, L. N. Bogatskaya, N. S. Verkhratsky, V. P. Zamostian, V. G. Shevtchuk, and I. V. Shtchegoleva (1970) Catecholamines in the metabolism and functions regulation in aging. Gerontologia. 16: 129-140.

Fuhr, A., R. E. Johnson, H. Kaunitz, and C. A. Slanetz (1949) Increased tocopherol requirements during the rat's menopause. Ann. N.Y. Acad. Sci. 52: 83-87.

Fujisawa, H. (1969) Changes in the isozyme patterns of lactate dehydrogenase (LDH) in the in vitro cultured muscle cells. Embryologia 10(3-4): 256-272.

% References from page 101 image

Glenister, T. W. (1970) Ovo-implantation in vitro and its relation to normal implantation. In: ``Ovo-Implantation. Human Gonadotropins and Prolactin'', pp. 73-85. Karger, N. Y.

Goldfischer, S. and J. Bernstein (1969) Lipofuscin (aging) pigment granules of the newborn human liver. J. Cell Biol. 42: 253.

Gonzalez-Angulo, A., J. Rivadeneyra and J. Martinez-Manautou (1970) Mitochondrial deformities in liver cells after high doses of estrogens. J. Cell Biol. 47: 74.

Goodfriend, T. L. and N. O. Kaplan (1964) Effects of hormone administration on lactic dehydrogenase. J. Biol. Chem. 239(1): 130-135.

Goodland, R. L., J. G. Reynolds, A. B. McCoord, W. T. Pommerenke (1953) Respiratory and electrolyte effects induced by estrogen and progesterone. Fert. and Ster. 6(4): 300-317.

Gram, T. E. and J. R. Fouts (1966) Effect of $\alpha$-tocopherol upon lipid peroxidation and drug metabolism in hepatic microsomes. Arch. Biochem. Biophys. 114: 331-335.

Greengard, O. (1967) Specific enzyme responses in starved, glucagon-treated or irradiated rats, and general consideration on the mechanisms of induction in animals. Adv. in Enz. Reg. 5: 397-405.

Greenwald, G. S. (1957) Interruption of pregnancy in the rabbit by the administration of estrogen. J. Exp. Zool. 135: 461.

Grinevich, M. A. and I. I. Breikman (1970) Investigation of complex formulations or oriental medicines and their components. Communication \#3: Tonic substances in the medicinal therapy of oriental medicine. Rastitelnyya Resursy 6(4): 481.

Gross, M. (1961) Reproductive cycle biochemistry. Fert. \& Ster. 12(3): 245.

Gurdon, J. B. (1962) The developmental capacity of nuclei taken from intestinal epithelium cells of feeding tadpoles. J. Embryol. Exp. Morph. 10: 622-640.

Gyorgy, P. and C. S. Rose (1949) Tocopherol and hemolysis in vivo and in vitro. Ann. N. Y. Acad. Sci. 52: 231.

% References from page 102 image

Hahn, H. F. and E. Fritz (1966) Age-related alterations in the structure of DNA: III. Thermal stability of rat liver DNA, related to age, histone content, and ionic strength. Gerontologia 12: 237.

Hahn, E. W. and R. L. Hays (1962) Modification of secondary sex ratio. J. Reprod. Fertil. 5: 409-411.

Hahn, E. W. and W. F. Ward (1969) The incidence of mating in X-irradiated female rats. J. Reprod. Fertil. 20: 151-154.

Harkness, M. L. R., R. D. Harkness and B. E. Moralee (1956) Loss of collagen from the uterus of the rat after ovariectomy and from the non-pregnant horn after parturition. Quart. J. Exp. Physiol. 41: 254-262.

Hazlewood, C. F., B. L. Nichols, D. C. Chang and B. Brown (1971) On the state of water in developing muscle: A study of the major phase of ordered water in skeletal muscle and its relationship to sodium concentration. The Johns Hopkins Med. J. 128(3): 117-131.

Henneman, D. H. (1968) Effect of estrogen on in vivo and in vitro collagen biosynthesis and maturation in old and young female guinea pigs. Endocrinol. 83: 678-690.

Hoffman, R. A. (1971) Influence of some endocrine glands, hormones and blinding on the histology and porphyrins of the Harderian glands of golden hamsters. Am. J. Anat. 132: 463-478.

Hollander, V. P. and M. L. Stephens (1959) Studies on phenol-activated oxidation of reduced nucleotides by rat uterus. J. Biol. Chem. 234(7): 1901-1906.

Horikoshi, H. and W. G. Wiest (1971) Interrelationships between estrogen and progesterone secretion and trauma-induced deciduomata. On causes of uterine refractoriness in the ``Farlow Rat.'' Endocrinol. 89: 807-817.

Houchin, O. B. (1942) The in vitro effect of $\alpha$-tocopherol and its phosphate derivative on oxidation in muscle tissue. J. Biol. Chem. 146: 313-321.

Ingram, D. L. and A. M. Mandl (1958) The hypophysial control of the X-ray sterilized ovary. J. Endocrin. 17: 1-12.

Jakubowska-Naziemblo, B. (1969) Effects of hypothalamic infusion of extract from pregnant uterus on the length of sexual cycle in guinea pigs. Acta Phys. Polonica XX(3): 47.

% References from page 103 image

Jeffery, J. J., R. J. Coffer and A. Z. Eisen (1971) Studies on uterine collagenase in tissue culture. II. Effect of steroid hormones on enzyme production. Biochim. et Biophys. Acta 252: 143-149.

Jellinck, P. H. and C. R. Lyttle (1971) Metabolism of 4-$^{14}$C-oestradiol by oestrogen-induced uterine peroxidase. Acta Endocrin. (Kbh) Suppl. 155: 122.

Jensen, E. V., M. Numata, S. Smith, T. Suzuki, P. I. Brecker, and E. R. DeSombre (1967) Estrogen-receptor interaction in target tissue. Dev. Biol. Suppl. 3: 151.

Jervell, K. F., G. R. Dintz, G. C. Mueller (1958) Early effects of estradiol on nucleic acid metabolism in the rat uterus. J. Biol. Chem. 231: 945.

Johansson, G. (1966) Influence of oxygen on the lactate dehydrogenase isozyme pattern in Chang liver cells. Exp. Cell Res. 43: 95-97.

Jones, A. W. (1972) Paper presented at N. Y. Acad. Sci. Intern. Conf. on Interactions of Ions and Water with Macromolecules and Biological Systems, Jan. 14, N. Y.

Jones, E. C. (1970) The ageing ovary and its influence on reproductive capacity. J. Reprod. Fertil. Suppl. 12: 17-30.

Jusiak, R. (1970) Possible sources of errors in determinations of tissue slices respiration. Acta Physiol. Pol. 21(4): 581-587.

Kao, K-Y. T., W. E. Hitt, A. T. Bush, and T. H. McGavack (1964) Connective tissue. XII. Stimulating effects of estrogens on collagen synthesis in rat uterine slices. Proc. Soc. Exp. Biol. Med. 117: 86-97.

Kar, A. B., V. P. Kamboj, A. P. Chowdhury (1966) Effect of a foreign body on oxygen consumption of the rat uterus. Curr. Sci. 35: 543.

Kar, A. B., V. P. Kamboj, A. Goswami and S. R. Chowdhury (1965) Effect of an intrauterine contraceptive suture on the uterus and fertility of rats. J. Reprod. Fert. 9: 317.

Karkun, J. N., B. Malviya and A. B. Kar (1969) Effect of intrauterine contraceptive suture on biochemical composition of the rat uterus during implantation. Indian J. Exp. Biol. 7(2): 118.

Karnaukhov, V. N. (1971) Role of carotenoids in intracellular deposition of oxygen. Doklady Biophys. 196: 13.

% References from page 104 image

Kaye, A. M., I. Icekson, H. R. Lindner (1971) Stimulation by estrogens of ornithine and S-adenosylmethionine decarboxylases in the immature rat uterus. Biochim. Biophys. Acta 252: 150-159.

Katsh, S., and A. Edelmann (1964) The influence of gonadectomy, sex, and strain upon the survival times of X-irradiated mice. In: ``Effects of Ionizing Radiation on the Reproductive System'' (W. D. Carlson and F. X. Gassner, eds.), p. 427. MacMillan, N. Y.

Kaunitz, H. and C. A. Slanetz (1947) Influence of alpha tocopherol on implantation in old rats. Proc. Soc. Exp. Biol. Med. 66: 334.

Kaunitz, H. and C. A. Slanetz (1948) Implantation in normal and vitamin E deficient rats. J. Nutr. 36: 331.

Kaunitz, H., C. A. Slanetz, and W. B. Atkinson (1948) Estrogen response and pigmentation of the uterus in Vitamin E-deficient rat. Proc. Soc. Exp. Biol. Med. 70: 302-304.

Kennedy, G. Y. (1970) Harderoporphyrin: A new porphyrin from the harderian glands of the rat. Comp. Biochem. Physiol. 36: 21-36.

Kimeldorf, D. J. and A. L. Soderwall (1947) Changes in the adrenal cortical zones by ovarian hormones. Endocrinol. 41(1): 21-26.

Kirkpatrick, J. F. (1971) Differential sensitivity of preimplantation mouse embryos in vitro to oestradiol and progesterone. J. Reprod. Fert. 27: 283-285.

Kobara, J. J. and D. Konvich (1972) The extraction of brain isozymes with solvents of varying polarity. Proc. Soc. Exp. Biol. Med. 139(4): 1326-1333.

Kohn, R. R. (1971) ``Principles of Mammalian Aging.'' Prentice Hall.

Kohn, R. R. (1971) Effects of antioxidants on life-span of C57BL mice. J. Gerontol. 26(3): 378-380.

Korenchevsky, V. (1961)``Physiological and Pathological Ageing.'' (G. H. Bourne, ed.) Hafner, N. Y.

Koszalka, T. R., and G. L. Andrew (1972) Effect of insulin on the uptake of creatine-1-$^{14}$C by skeletal muscle in normal and X-irradiated rats. Proc. Soc. Exp. Biol. Med. 139(4): 1265-1271.

% References from page 105 image

Kraay, R. J. and L. J. Black (1970) The effect of estrogen priming on the uptake of radioactive estradiol. Proc. Soc. Exp. Biol. Med. 133: 376-379.

Kreutzer, H. H. and W. H. S. Fennis (1964) Lactic dehydrogenase isozymes in blood serum after storage at different temperatures. Clin. Chim. Acta 9: 64-68.

Krohn, P. L. (1966) Transplantation and aging. In: ``Topics in the Biology of Aging.'' (P. L. Krohn, ed.), p. 125. Wiley and Sons, N. Y.

Labhsetwar, A. P. (1970) Ageing changes in pituitary ovarian relationships. J. Reprod. Fertil. Suppl. 12.

Labhsetwar, A. P. and N. Perser (1972) Uterine uptake of 6,7-$^3$H oestradiol in the presence of intrauterine contraceptive device in cyclic and pregnant rats. Acta Endocrinol. 69: 583-588.

Lansing, A. (1947) A transmissable, cumulative, and reversible factor in aging. J. Gerontol. 2: 228-239.

Lazar, G. and G. L. Farkas (1970) Patterns of enzyme changes during leaf senescence. Acta Biol. Acad. Sci. Hung. 21(4): 389-396.

Leavit, W. W. and G. C. Blaha (1970) Circulating progesterone levels in the golden hamster during estrous cycle, pregnancy, and lactation. Biol. Reprod. 3: 353-361.

Lecoq, R. and P. Isidor (1949) Studies on the histopathology of vitamin E deficiency. Ann. N. Y. Acad. Sci. 52: 139-141.

Lecoq, R., P. Chauchard and H. Mazoue (1949) Action of vitamin E on the non-deficient organism. Ann. N. Y. Acad. Sci. 52(3): 142.

Lee, C. and H. I. Jacobson (1971) Uterine estrogen receptor in rats during pubescence and the estrous cycle. Endocrinol. 88: 596.

Lee, Y. C. P., J. T. King, and M. B. Visscher (1952) Influence of protein and calorie intake upon certain reproductive functions and carcinogenesis in the mouse. Amer. J. Physiol. 168(2): 391-399.

Lehmann, F.,G. Bettendorf and Ch. Neale (1971) Plasma steroid patterns in normal, pathological and induced ovarian cycles. Acta Endocrin. (Kby.) Suppl. 155: 76.

% 10

Lehninger, A. L. (1970) ``Biochemistry.'' Worth, N. Y.

Lezzi, M. (1969) Differential effects of sodium and potassium ions on the template activity of rat liver chromatin. Physiol. Chem. Phys. 1: 447.

Levvy, G. X., L. M. H. Kerr, and J. G. Campbell (1948) $\beta$-Glucuron-idase and cell proliferation. Biochem. J. 42: 462.

Lilly, J. C. ``The Dolphin in History.'' By J. C. Lilly and A. Montagu, University of California, L. A. (1963).

Ling, G. N. (1972) Paper presented at N. Y. Acad. of Sci. Intern. Conf. on Interactions of Ions and Water with Macromolecules and Biological Systems, Jan. 14, N. Y.

Ling, G. N. and M. H. Kromash (1967) The extracellular space of voluntary muscle tissues. J. Gen. Physiol. 50(3): 677-694.

Loeb, L., V. Suntzeff and E. L. Burns (1939) Changes in the nature of the stroma in vagina, cervix and uterus of the mouse produced by long continued injections of oestrogen and by advancing age. Amer. J. Cancer 35: 159.

Loewenstein, J. M. (1972) Ammonia production in muscle and other tissues: The purine nucleotide cycle. Physiol. Rev. 52(2): 382.

Longmuir, I. S., J. Pistel, I. L. Magnus (1958) Respiration of epidermal cells. Bioch. Biophys. Acta 29: 203.

Longmuir, I. S. (1957) Respiration of rat-liver cells at low oxygen concentrations. Biochem. J. 65: 378.

Lucas, F. V., H. A. Neufeld, J. G. Utterback, A. P. Martin, and E. Stotz (1955) The effect of estrogen on the production of a peroxidase in the rat uterus. J. Biol. Chem. 214: 775-780.

Lunaas, T. (1964) A simple method for detection of urinary estrogens. Zootecnia e veterinari 14: 359-364.

Maharajh, D. M. and J. Walkley (1972) Lowering of the saturation solubility of oxygen by the presence of another gas. Nature 236: 165.

Mahmoud, I. Y. and J. Klicka (1971) Developmental changes in lactic dehydrogenase isozymes in turtles. Life Sci. Pt. II, 10(16): 955.

Malaviya, B., J. N. Karkun and A. B. Kar (1970) Effect of intrauterine contraceptive suture on the response of rat uterus to prolonged estrogen treatment. Indian J. Exp. Biol. 8: 19.

McCay, P. B., J. L. Poyer, P. M. Pfeifer, H. E. May (1971) A function of $\alpha$-tocopherol: stabilization of the microsomal membrane from radical attack during TPNH-dependent oxidations. Lipids 6(5): 297-306.

McGuire, J. L., G. D. Bariso, B. S. Fuller (1972) Preliminary isolation of an estrogen specific binding macromolecule from the human uterus. J. Clin. Endocr. Metab. 34(1): 243-246.

McLaren, A. and T. M. Menke (1971) CO$_2$ output of mouse blastocysts in vitro, in normal pregnancy and in delay. J. Reprod. Fert. suppl. 14: 23-29.

Martin, L. (1963) Interactions between oestradiol and progesterone in the uterus of the mouse. J. Endocrin. 26: 31-39.

Martin, L. (1960) The use of 2-3-5-triphenyltetrazolium chloride in the biological assay of oestrogens. J. Endocrin. 20: 187-197.

Martin, L., C. A. Finn and J. Carter (1970) Effects of progesterone and oestradiol-17$\beta$ on the luminal epithelium of the mouse uterus. J. Reprod. Fert. 21: 461-469.

Mathew, G. D. (1949) Pelvic endometriosis in relation to sterility. Studies on Fertility \# 1: 18.

Matusis, L. I. (1969) Effect of $\alpha$-tocopherol on creatine kinase activity of skeletal muscles and blood in primary and secondary vitamin K deficiency. Bull. Exp. Biol. Med. 69(3): 282.

Matusis, L. I. (1971) Effect of administration of $\alpha$-tocopherol to albino rats on changes in content of ATP, ADP, and inorganic phosphorus in the skin and skeletal muscles due to avitaminosis K. Bull. Exp. Biol. Med. 71(5): 542-543.

Maurer, R. R. and R. H. Foote (1971) Maternal ageing and embryonic mortality in the rabbit. J. Reprod. Fert. 25: 329-341.

Meadow, N. D. and G. H. Barrows, Jr. (1971) Studies on aging in a bdelloid rotifer. II. The effects of various environmental conditions and maternal age on longevity and fecundity. J. Gerontol. 26(3): 302-309.

Mestwerdt, W., H. Brandau and H. H. Kley (1972) Struktur und Funktion Steroidaktiver zellen im Postmenopauseovar. Acta Endocr. (Kbh) Suppl. 159: 13.

Metchnikof, E. (1908) ``The Prolongation of Life.'' G. P. Putman's Sons, N. Y.

Mitchell, J. A. and J. M. Yochim (1968) Intrauterine oxygen tension during the estrous cycle in the rat: its relation to uterine respiration and vascular activity. Endocrinol. 83: 701-705.

Mitchell, J. A. and J. M. Yochim (1968) Measurement of intrauterine oxygen tension in the rat and its regulation by ovarian steroid hormones. Endocrinol. 83: 691-700.

Mole, R. H. (1963) Does radiation age or produce non-specific life-shortening? In: ``Cellular Basis and Aetiology of Late Somatic Effects of Ionizing Radiation'' (R. J. C. Harris, ed.), p. 273. Academic Press, N. Y.

Morgulis, S. and W. Osheroff (1938) Mineral composition of the muscles of rabbits on a diet producing muscle dystrophy. J. Biol. Chem. May, pp. 767-773.

Mueller, G. C., A. M. Herranen and K. F. Jervell (1958) Studies on the mechanism of action of estrogens. Rec. Prog. in Hormone Res. XIV: 95-139.

Needham, A. E. (1964) ``The Growth Process in Animals.'' Van Nostrand Co., Princeton.

Needham, J. (1931) ``Chemical Embryology.'' Cambridge University Press, Cambridge.

Needham, J. (1942) ``Biochemistry and Morphogenesis.'' Cambridge University Press, Cambridge.

Needham, J. and Lu Gwei-Djeh (1968) Steroid Hormones in the Middle Ages. Endeavor 27: 130-132.

Nemetschek-Gansler, H. (1967) Ultrastructure of the myometrium. In: ``Cellular Biology of the Uterus'' (R. M. Winn, ed.), p. 353. Appleton-Century-Crofts, N. Y.

Nevo, A. C. (1965) Dependence of sperm motility and respiration on oxygen concentration. J. Reprod. Fert. 9: 103-107.

Nikolayev, A. I., F. B. Subkhankulova and I. S. Geller (1970) Immune reactions during intoxication with the pesticides methylmercaptophos, phosphamide, aldrin, and monuron. Farmakologiya i Toksikologiya 33(6): 737.

Olson, R. E. (1963) `Excess lactate' and anaerobiosis. Ann. Intern. Med. 59: 960-963.

Orr, C. W., M. Yoshikawa-fukada and J. D. Ebert (1972) Potassium: Effect on DNA synthesis and multiplication of baby-hamster kidney cells. Proc. Nat. Acad. Sci. USA 69(1): 243-247.

Orsini, M. W. (1961) The external vaginal phenomena characterizing the stages of the estrus cycle, pregnancy, pseudopregnancy, lactation and the anestrous hamster, \textit{Mesocricetus auratus} Waterhouse. Proc. Anim. Care Panel 11: 193-206.

Orsini, M. W. and R. K. Meyer (1962) Effect of varying doses of progesterone on implantation in the ovariectomized hamster. Proc. Soc. Exp. Biol. Med. 110: 713.

Otel, V., N. Arvanitopol and C. Situlescu (1968) Efectul progesteroului in infertilitatea Functionale la Vaci. Lucr. Stiint Inst. Cercet Zootech. 26: 113-118.

Palladin, A. V. (1964) In:``Problems of the Biochemistry of the Nervous System.'' Pergamon Press, N. Y.

P'an, S. Y., H. Kaunitz and C. A. Slanetz (1949) Restoration of vaginal estrus by alpha-tocopherol acetate in old rats. Ann. N. Y. Acad. Sci. 52: 80-82.

P'an, S. Y., H. B. van Dyke, H. Kaunitz and C. A. Slanetz (1949) Effect of vitamin-E deficiency on amount of gonadotrophin in the anterior pituitary of rats. Soc. Exp. Biol. Med. 72(3): 523-526.

Parkening, T. A. and A. L. Soderwall (1972) Delayed Embryonic development and implantation in senescent golden hamsters. In Press.

Patnaik, B. N. (1971) Age related studies on ascorbic acid metabolism. Gerontologia 17(2): 112.

Peat, R. and A. L. Soderwall (1971) Cold-inactivated enzymes as metabolic controls. Physiol. Chem. Phys. 3: 401-402.

Peat, R. and A. L. Soderwall (1972) Estrogen stimulated pathway changes and cold-inactivated enzymes. Physiol. Chem. Phys., In Press.

Pepe, G. J. and J. M. Yochim (1971) Glucose-6-phosphate dehydrogenase activity in the endometrium and myometrium of the rat uterus during the estrous cycle and progestation. Biol. Repr. 6: 161.

Pfeiffer, E., A. Verwoerdt and Hsioh-Shan Wang (1968) Geriatric Focus 7(13): 1-6.

Pincus, G. and R. E. Kirsch (1936) The sterility in rabbits produced by injections of oestrone and related compounds. Amer. J. Physiol. 115: 219.

Poulson, E., J. M. Robson, and F. M. Sullivan (1963) Embryopathic effects of progesterone deficiency. J. Endocr. 31: xxviii.

Pous, D. L. and J. Comas (1959) La Insuficiencia Progestacional del Endometrio. Proc. 3rd World Congr. Fert. Ster. (Amsterdam): 593-595.

Raab, W., F. van Lith, E. Lepeschkin, H. C. Herrlich (1962) Catecholamine-induced myocardial hypoxia in the presence of impaired coronary dilatability independent of external cardiac work. Amer. J. Cardiol. 9: 455-470.

Racker, E. (1972) Bioenergetics and the problem of tumor growth. Amer. Sci. 60: 56-63.

Reid, E. (1965) ``Biochemical Approaches to Cancer.'' Pergamon Press, N. Y.

Reichel, F. J. (1967) ``Concepts in Biochemistry.'' McGraw-Hill, N. Y.

Roberts, S. and C. M. Szego (1953) The influence of steroids on uterine respiration and glycolysis. J. Biol. Chem. 201: 21.

Rokitanskiy, V. I. and Yu. V. Yakimov (1971) Proton relaxation in joint tissue after trauma: Study based on nuclear magnetic resonance data. Ortopediya, Traumatologiya i Protezirovaniye 9: 35.

Rowlatt, C. (1970) Some effects of age and castration in the epithelial basal lamina of secondary sex organs in the mouse. Gerontologia (Basel) 16(3): 182.

Rusin, V. Ya. (1971) Results of a post-mortem study of endocrine glands and other organs of animals exposed to muscle training, cold adaptation or treated with dibazol. Biologicheskie Nauki, 4: 33.

Sacher, G. A. (1959) Relation of lifespan to brain and body weight. In: Ciba Foundation Coll. on Ageing, 5: 115.

Sacher, G. A. (1968) Molecular versus systemic theories on the genesis of ageing. Exp. Gerontol. 3: 265.

Saldarini, J. and J. M. Yochim (1967) Metabolism of the uterus of the rat during early pseudopregnancy and its regulation by estrogen and progesterone. Endocrinol. 80: 453-466.

Sanwal, P. C., J. K. Pande, P. R. Dasgupta, A. B. Kar and B. S. Setty (1970) Long-term effect of a continuous low dose of megestrol acetate on the genital organs and fertility of female rats. Steroids 15(5): 711.

Schaub, M. C. (1964/5) Changes of collagen in the aging and in the pregnant uterus of white rats. Gerontol. 10: 137.

Schmidt, H., I. Noack, H. Walther, and K. D. Voigt (1967) Einfluss von Cyclus und exogener Hormonzufuhr auf Stoffwechselvorgange des Rattenuterus. Acta Endocrinol. 56: 231-243.

Schultze, A. B. (1968) Selective breeding for uterine metabolic status and its relation to body size in rats. Growth 32: 163-164.

Schultze, A. B. (1967) Uterine metabolism changes during gestation in rats of different age groups. Proc. Soc. Exp. Biol. Med. 125: 379-382.

Schultze, A. B. (1965) Offspring of multiparous rats. J. Reprod. Fert. 10: 145-147.

Schultze, A. B. (1969) Results of breeding for low and high uterine metabolic status in rats. J. Hered. 60: 349-350.

Schultze, A. B. (1964) Triphenyltetrazolium reduction by uterine tissue of rats. Proc. Soc. Exp. Biol. Med. 116: 653.

Schultze, A. B. (1964) Changes in uterine metabolism associated with reproductive status in female rats. Proc. 5th Intern. Congr. Anim. Reprod. and Artif. Insemin. 2: 301.

Schwarz, K. (1949) Dietetic hepatic injuries and the mode of action of tocopherol. Ann. N. Y. Acad. Sci. 52: 225.

Senger, P. L., E. D. Lose, and L. C. Ulberg (1967) Reduced blood supply to the uterus as a cause for early embryonic death in the mouse. J. Exp. Zool. 165: 337-344.

Shain, S. A. and A. Barnea (1971) Some characteristics of the estradiol binding protein of the mature, intact rat uterus. Endocrinol. 89: 1270-1279.

Sharp, J. G., C. J. Papariello (1971) The effects of microwave exposure on thymidine-$^3$H uptake in albino rats. Radiation Res. 45: 434-439.

Shereshevsky, J. L. (1928) A study of vapor pressures in small capillaries. I. Water vapor. (A) Soft glass capillaries. J. Amer. Chem. Soc. 50: 2966.

Shereshevsky, J. L. and C. P. Carter (1950) Liquid-vapor equilibrium in microscopic capillaries. J. Amer. Chem. Soc. 72: 3682.

Singhal, R. L., J. R. E. Valadares, and G. M. Ling (1969) Estrogenic regulation of uterine carbohydrate metabolism during senescence. Am. J. Physiol. 217: 793-797.

Shock, N. W., M. J. Yiengst, J. A. Falzone (1963) Age differences in the water content of the body as related to basal oxygen consumption in males. J. Geront. 18: 1.

Shukla, S. P. and M. S. Kanungo (1970) Oxidation of pyruvate and lactate by the liver and brain homogenates of rats of various ages. Exp. Geront. 5: 171-175.

Shute, E. V. (1940) A reply to recent criticisms of the theory of a relationship between vitamin E and the oestrogens. J. Endocr. 2: 173-178.

Smith, M. G. (1926) Interruption of pregnancy in rat by injection of ovarian follicular extract. Bull. John Hopkins Hosp. 39: 203-214.

Snedcor, G. W. (1956) ``Statistical Methods.'' 5th Edn. Iowa State College Press, Ames.

Soderwall, A. L., H. A. Kent, C. L. Turbyfill and A. L. Britenbaker (1960) Variation in gestation length and litter size of the golden hamster, \textit{Mesocricetus auratus}. J. Geront. 15: 246-248.

Soderwall, A. L. and B. C. Smith (1962) Beneficial effect of wheat germ oil on pregnancies in female golden hamsters (\textit{Mesocricetus auratus}: Waterhouse). Fert. Steril. 13: 287.

Song, C. S. and A. Kappas (1968) The influence of estrogens, progestins, and pregnancy on the liver. Vitamins and Hormones 26: 147.

Spaziani, E. and R. P. Suddick (1967) Hexose transport and blood flow rate in the uterus: effects of estradiol, puromycin, and actinomycin D. Endocrinol. 81: 205-212.

Stambaugh, R. and J. Buckley (1971) Association of the lactic dehydrogenase X$_4$ isozyme with male-producing rabbit spermatozoa. J. Reprod. Fert. 25: 275-278.

Steinach, E. (1940) ``Sex and Life.'' Viking Press, N. Y.

Stockton, B. (1972) Reproductive senescence in the golden hamster, \textit{Mesocricetus auratus} Waterhouse: Temporal aspects of ovulation and induction of uterine decidual cells. Ph. D. Dissertation, University of Oregon, Eugene.

Strehler, B. L. (1967) In: Soc. Exp. Biol. Symp. XXI, Aspects of the Biology of Aging, (H. W. Woolhouse, ed.), p. 149. Academic Press, N. Y.

Strickland, K. P., R. H. S. Thompson and G. R. Webster (1956) The effects of anticholinesterases on oxidative phosphorylation in brain, and their relation to potassium loss and nerve degeneration. Biochem. J. 62: 512.

Strong, L. C. (1942) Sex differences in pigment content of Harderian glands of mice. Proc. Soc. Exp. Biol. Med. 50: 123.

Strong, L. C. (1968) ``Biological Aspects of Cancer and Aging.'' Pergamon Press, N. Y.

Strong, L. C., F. N. Johnson and A. A. Rimm (1965) The effects of fifty-four generations of inbreeding on age of first litters. J. Gerontol. 20(3): 405-409.

Surani, M. A. H. and P. J. Heald (1971) The metabolism of glucose by rat uterus tissue in early pregnancy. Acta Endocrinol. 66: 16-24.

Takeuchi, T., M. Hayashi and W. H. Fishman (1962) Mucinogenesis and enzymorphology of rat vagina: phosphorylase, $\beta$-glucuronidase, DPNH-diphorase and phosphatases. Acta Endocrinol. 39: 395-406.

Talbert, G. B. and P. L. Krohn (1966) Effect of maternal age on viability of ova and uterine support of pregnancy in mice. J. Reprod. Fert. 11: 399-406.

Talbert, G. B. (1971) Effects of maternal age on post-implantation reproductive failure in mice. J. Reprod. Fert. 24: 449.

Talalay, P., W. H. Fishman, and C. Huggins (1946) Chromogenic substrates. II. Phenolphthalein glucuronic acid as substrate for the assay of glucuronidase activity. J. Biol. Chem. 166: 757.

Talalay, P. and H. G. Williams-Ashman (1958) Activation of H-transfer between pyridine nucleotides by steroid hormones. Proc. Natl. Acad. Sci. 4: 15.

Tart, C. C. (1972) States of consciousness and state-specific sciences. Science 176.

Temple, S., V. P. Hollander, N. Hollander, and M. L. Stephens (1960) Estradiol activation of uterine reduced diphosphopyridine nucleotide oxidase. J. Biol. Chem. 235(5): 1504-1509.

Thorneycroft, I. H. and A. L. Soderwall (1969) The nature of the litter size loss in senescent hamsters. Anat. Rec. 165: 343-348.

Thorneycroft, I. H. and A. L. Soderwall (1969) Ovarian morphological and functional changes in reproductively senescent hamsters. Anat. Rec. 165: 349-354.

Thung, P. J. and C. F. Hollander (1967) Regenerative growth and accelerated ageing. In: Soc. Exp. Biol. Symp. XXI, Aspects of the Biology of Aging, (H. W. Woolhouse, ed.). Academic Press, N. Y.

Tobin, R. B., M. A. Mehlman (1971) pH Effects on O$_2$ consumption and on lactate and pyruvate production by liver slices. Amer. J. Physiol. 221(4): 1151-1155.

Tomkins, G. M., K. L. Yielding, and J. Curran (1961) Steroid hormone activation of L-alanine oxidation catalyzed by a subunit of crystalline glutamic dehydrogenase. Proc. Natl. Acad. Sci. 47: 270-278.

Tomlinson, R. W. S. (1971) The action of progesterone on the sodium transport of isolated frog skin. Acta Physiol. Scand. 83: 463-472.

Trams, G., B. Engel, F. Lehmann, and H. Maass (1971) Specific binding of oestradiol in human uterine tissue. Acta Endocrin. (Kbh.) Suppl. 155: 138.

Troshin, A. S. (1966) ``Problems of Cell Permeability.'' Pergamon Press, N. Y.

Ungar, G., E. Aschheim, S. Psychoyos, and D. V. Romano (1957) Reversible changes of protein configuration in stimulated nerve structures. J. Gen. Physiol. 40: 635.

Ungar, G. and S. Kadis (1959) Effect of insulin on sulphydryl groups in muscle. Nature 183(4653): 49-50.

Valentini, E. J. and E. W. Hahn (1971) The indirect effect of radiation on embryonic mortality. Int. J. Radiat. Biol. 20(3): 259-267.

Verzar, F., and M. Ermini (1970) Decrease of creatine-phosphate restitution of muscle in old age and the influence of glucose. Gerontol. 16: 223-230.

Villee, C. A., D. D. Hagerman, and P. B. Joel (1965) An enzymatic basis for the physiologic function of estrogens. Rec. Prog. Hormone Res. 16: 46-69.

Vinson, G. P. and I. C. Jones (1964) The in vitro production of oestrogens from progesterone by mouse adrenal glands. J. Endocrinol. 29: 185-191.

Visscher, M. B., J. T. King, and Y. C. P. Lee (1952) Further studies on influence of age and diet upon reproductive senescence in strain A female mice. Amer. J. Physiol. 170: 72.

Waddell, W. J. and R. G. Bates (1969) Intracellular pH. Physiol. Rev. 49: 285.

Wakabayashi, M. (1970) $\beta$-glucuronidases in metabolic hydrolysis. In: ``Metabolic Conjugation and Metabolic Hydrolysis,'' Vol. II (W. H. Fishman, ed.), pp. 520-602. Academic Press, N. Y.

Warburg, O. (1969) ``The Prime Cause and Prevention of Cancer.'' Konrad Triltsch, Wurzburg, Germany.

Warburg, O., A. W. Geissler, and S. Lorenz (1967) Oxygen, the creator of differentiation. In: ``Aspects of Yeast Metabolism'' (A. K. Mills, ed.), p. 327. Blackwell, Oxford.

Webb, S. J. (1965) ``Bound Water in Biological Integrity.'' Thomas, Springfield.

Wiggins, P. M. (1971) Water structure as a determinant of ion distribution in living tissue. J. Theor. Biol. 32: 131-146.

Wigglesworth, J. S. (1964) Experimental growth retardation in the rat. J. Pathol. Bacteriol. 88(1):1.

Wilson, E. W. (1969) The effect of oestradiol-17$\beta$ on enzymes concerned with metabolism of carbohydrate in human endometrium in vitro. J. Endocr. 44: 63-68.

Wintenberger-Torres, S. and M-C. Danel (1971) Etude du developpement in vitro des blastocystes de lapin. Ann. Biol. Anim. Bioch. Biophys. 11(3): 379-387.

With, T. G. (1968) ``Bile Pigments.'' Academic Press, N. Y.

Woessner, J. F. (1962) Ageing of human uterus connective tissue. J. Gerontol. 17: 453.

Yanagishima, N., C. Shimoda, T. Takahashi, and N. Takao (1970) Responsiveness of yeast cells to auxin, animal sex hormones and yeast sexual hormones in relation to sex controlling genes. Devel. Growth, Different. 11(4): 277-285.

Yochim, J. M. and J. A. Mitchell (1968) Intrauterine oxygen tension in the rat during progestation: its possible relation to carbohydrate metabolism and the regulation of nidation. Endocrinol. 83: 706-713.

Yochim, J. M. (1971) Intrauterine oxygen tension and the metabolism of the endometrium during the preimplantation period. In: ``Biology of the Blastocyst'' (R. J. Blandau, ed.), p. 363. University of Chicago Press, Chicago.

Yochim, J. M. and S. W. Clark (1971) Lactic dehydrogenase activity in the uterus of the rat during the estrous cycle and its relation to intrauterine oxygen tension. Biol. Repr. 146-151.

Yokota, K. and Yamazaki (1965) Reaction of peroxidase with reduced nicotinamide-adenine dinucleotide and reduced nicotinamide-adenine dinucleotide phosphate. Biochim. Biophys. Acta 105: 301-312.

Yudkin, F. (1970) A rival of the legendary ginseng. Ural 11: 77-79.

Zinkat, W. (1931) Hist-topochemische untersuchungen uber die Schwankungen des Kalkge haltes den Arterien des Uterus. Virchow's Arch. Path. Anat. u. Physiol. 281(3): 911-931.

Zondag, H. A. (1963) Lactate dehydrogenase isozymes: lability at low temperature. Nature 143: 965-967.